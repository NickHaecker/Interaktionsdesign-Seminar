% Holodecks template version of 2023-09-01 enhances the ACM template, version 1.7.0:
% https://www.acm.org/publications/proceedings-template
% The ACM Latex guide provides further information about the ACM template

\documentclass[sigconf, nonacm]{acmart}
\usepackage[german]{babel}
\usepackage{csquotes}
%% The following content must be adapted for the final version
% paper-specific
\newcommand\holodoi{XX.XX/XXX.XX}
\newcommand\holopages{XXX-XXX}
% issue-specific
\newcommand\holovolume{1}
\newcommand\holoissue{1}
\newcommand\holoyear{2023}
% should be fine as it is
\newcommand\holoauthors{\authors}
\newcommand\holotitle{\shorttitle} 
% leave empty if no availability url should be set
\newcommand\holoavailabilityurl{URL_TO_YOUR_ARTIFACTS}
% whether page numbers should be shown or not, use 'plain' for review versions, 'empty' for camera ready
\newcommand\holopagestyle{plain} 

\begin{document}
\title{Konzeption eines Projektmanagement Task-Boards unter Anwendung des Habitica Gamification-Ansatzes}

%%
%% The "author" command and its associated commands are used to define the authors and their affiliations.
\author{Nick Philipp Häcker}
\affiliation{%
  \institution{Fakultät für Digitale Medien}
  \institution{Hochschule Furtwangen University}
}
\email{nick.athaeck@gmail.com}




%%
%% The abstract is a short summary of the work to be presented in the
%% article.
\begin{abstract}
In der heutigen Welt stehen viele sowohl Studierende als auch Werktätige vor einer Vielzahl von Herausforderungen, um ihre Produktivität aufrechtzuerhalten. Um das zu erreichen gibt es viele verschiedene Möglichkeiten. Dabei werden Aufgaben häufig in Papier Format aufgeschrieben, oder es werden Tools wie Notion, GitHub Projects oder Habitica verwendet. 
Diese Arbeit beschäftigt sich mit dem Gamification-Ansatz von Habitica, sammelt seine Schwachstellen und konzipiert Lösungen für eine bessere Integration der Spielerischen Elemente. Dabei wird in der Konzeption des Aufgaben-Boards der Aufbau und die Funktionsweise eines gestaltbaren Projektmanagements Tools, wie er in Asana vorzufinden ist, miteinbezogen und angewandt.\\
(nochmal überarbeiten um das gesamte Paper am Ende zusammenfassen zu können)
\end{abstract}

\maketitle

%%% do not modify the following Holodeck block %%
%%% Holodeck block start %%%
\pagestyle{\holopagestyle}
\begingroup\small\noindent\raggedright\textbf{CCS Concepts:}\\
CCS $\rightarrow$ Human-centered computing $\rightarrow$ Human computer interaction (HCI) $\rightarrow$ HCI design and evaluation methods $\rightarrow$ User studies;
CCS $\rightarrow$ Theory of computation $\rightarrow$ Theory and algorithms for application domains $\rightarrow$ Algorithmic game theory and mechanism design $\rightarrow$ Network games; CCS $\rightarrow$ Software and its engineering $\rightarrow$ Software organization and properties $\rightarrow$ Contextual software domains $\rightarrow$ Virtual worlds software $\rightarrow$ Interactive games


 \renewcommand\thefootnote{}\footnote{\noindent
	Seminar Interaktionsdesign \\
	\emph{Interaktionsdesign MIM1}, WS 2023/2024\\
	Dozent: Prof. Dr. Thomas Schlegel\\
	Modul: Interaktionsdesign, Fakultät Digitale Medien
 \\
 }\addtocounter{footnote}{-1}\endgroup
%%% Holodecks block end %%%

%%% do not modify the following Holodecks block %%
%%% Holodecks block start %%%
\ifdefempty{\holoavailabilityurl}{}{
\vspace{.3cm}
\begingroup\small\noindent\raggedright\textbf{Keywords}\\
Gamification, Asana, 3D Environment, Task-Management
% The source code, data, and/or other artifacts have been made available at \url{\holoavailabilityurl}.
\endgroup
}
%%% Holodecks block end %%%

\section{Einleitung}
Das in der Veranstaltung \enquote{Interaktionsdesigin} entstandene Projekt der Hochschule Furtwangen dient als technisches \enquote{prove-of-concept} um eine Anwendung, wie sie hier konzipiert wird, umzusetzen.  \\
Ein bevorzugtes Mittel um seine Tages-, Wochen- oder gar Monatsaufgaben zu sammeln und zu planen, greifen viele Menschen zu Stift und einem Blatt Papier oder einem Kalender und schreiben sich Termine und/ oder Aufgaben auf. Für die Einen dient ein Kalender und ein Paper oder Block als ToDo-Liste, für die Anderen dienen dabei digitale Anwendungen als Hilfsmittel für die Aufgabenplanung.\\
Um einen Projektplaner nicht nur Monoton und dem Nutzen zum Zweck zu gestalten, gab es die Idee Gamification-Elemente einzubinden um eine Mischung aus einem ToDo-Board und einem kleinem "Role-Playing-Game" (RPG) zu erhalten. Dieser Idee ist bereits ein Entwicklerteam gefolgt und haben eine Anwendung gebaut, die diese 2 Aspekte vereint. Es dient zusätzlich als ein "Habit-Tracker", um den eigenen Lebensstil anzupassen. Dieses Paper bedient sich dabei an einer Forschung der Ludwig-Maximilians-Universität, München, Fakultät der Psychologie mit dem Namen \enquote{Counterproductive effetcs of gamification: An analysis \allowbreak on the example of the gamified task manager Habitica}. Dabei wurden bei einer Nutzerstudie von 45 Probanden Kontraproduktive Nebeneffekte festgestellt, welche in der Anwendung Habitica festgestellt wurde. Das Ziel ist es, diese Nebeneffekte zu sammeln und Lösungen zu konzipieren, die in einer neuen Form dieser Anwendung umgesetzt werden sollen. Dabei werden bestehende Projektmanagement-Tools und Gamification Ansätze betrachtet, die für die Konzeption relevant sind. Die Konzeption orientiert sich dabei am User-Centered-Design-Prozess nach ISO 9245-210 und wird in Form eines abgewandelten Nutzertests evaluiert..\\
Die Forschungsfrage dieses Paper lautet daher:\\
\textbf{Wie können paradoxe Nebeneffekte der Bedienung \allowbreak von Habitica durch das Miteinbeziehen eines Projektmanagement Tools wie Asana vermindert werden?}
\\
\\
\\
(Hier Struktur des Papers einbauen, sobald alle Kapitel geschrieben sind)
\section{Grundlagen}
\subsection{Gamification}

\subsubsection{Gamification Elemente}

\subsubsection{Spielertypen nach Bartle}

\subsubsection{Motivation}

\subsection{Projektmanagement und Tools im Allgemeinen}
Das Wort Projektmanagement besteht aus den zwei Wörtern \enquote{Projekt} und \enquote{Management}. 
In diesem Kontext wird ein Projekt nach der DIN 69901-5 als eine \enquote{Absicht, die im Wesentlichen durch die Einzigartigkeit der Bedienung in ihrer Gesamtheit gekennzeichnet ist} Es zeichnet sich durch eine Zielvorgabe mit zeitlichen, finanziellen und personellen Einschränkungen aus.~\cite{DIN69901-5} 
Der Begriff \enquote{Management} \enquote{bezeichnet heute im betriebswirtschaftlichen Sprachgebrauch einerseits - in funktionaler Perspektive - die Tätigkeit der Unternehmensführung. Andererseits wird auch - in institutioneller Perspektive - das geschäftsführende Organ, also die Gruppe der leitenden Personen eines Unternehmens als Management bezeichnet.}~\cite{haric_definition_nodate} 
Die Managementaufgaben umfassen die Definition von Organisationszielen, die Ausarbeitung einer Strategie zur Zielerreichung, sowie die Organisation und Koordination der Produktionsfaktoren sowie die Führung der Mitarbeiter.
Kombiniert bezeichnet der Begriff \enquote{Projektmanagement} die \enquote{Gesamtheit von Führungsaufgaben, -organisation, -techniken und -mitteln für die Initiierung, Definition, Planung, Steuerung und den Abschluss von Projekten}~\cite{DIN69901-5}
\\
Generell hilft das Projektmanagement einem Team bei der Organisation, Nachtverfolgung und Durchführung bei der Projektarbeit. Innerhalb des Projektmanagements gibt es verschiedene Arten, Methoden und Ansätze. Es gibt das Agile Projektmanagement, das Wasserfallmodell, die PRINCE2-Methode und die Methode des kritischen Pfades.~\cite{asana_projektmanagement_nodate} Da es das Ziel dieser Arbeit ist, den Mechanismus und Ablauf des Projektmanagements mit der Anwendung Habitica zu kombinieren, wird im Folgenden nur auf das Agile Projektmanagement eingegangen.

\subsubsection{Agiles Projektmanagement}
Das \enquote{Agile Projektmanagement beschreibt eine interaktive Methodik, bei der Arbeitsphasen in kurze Sprint aufgeteilt werden.}~\cite{asana_kanban_nodate} Für die Anwender ist dieser Ansatz am Flexibelsten, da sie in bestehenden kurzen Zeitabschnitten Zwischenergebnisse präsentieren können, und schnell auf Änderungen der Gegebenheiten reagieren können und diese Einplanen. Allerdings kann dadurch ungeplante Mehrarbeit entstehen.
%sollte man hier das manifest noch erwähnen?
%Für die Agile Entwicklung bzw. des agilen Managements wurde ein Manifest verfasst, welches aus 4 Werten und 12 Prinzipien besteht. Die Werte handeln davon, dass Individuen und Interaktionen mehr als Prozesse oder Werkzeuge sind. Eine funktionierende Software wichtiger ist als eine umfassende Dokumentation. Die Zusammenarbeit mit dem Kunden wichtiger ist, als die Vertragsverhandlung und dass das Reagieren auf Veränderungen höher priorisiert ist, als das Befolgen eines Plans.~\cite{noauthor_manifest_nodate}
%Aus diesen 
Das Agile-Projektmanagement basiert auf iterativen Prozessen, wie dem Backlog-Management, Sprints, Retrospektiven, Iterationen und weiteren Sprints.
Das Backlog ist eine Liste von Aufgaben, die während des Sprints bearbeitet werden können. Diese Aufgaben stehen in der Regel in einem Kanban Board, welches im folgenden Kapitel vorgestellt wird. Sprints dienen einem Zeitrahmen, in welchem eine bestimmte Anzahl von Aufgaben aus dem Backlog erledigt werden sollen. Um diese Aufgaben, die erledigt werden sollen zu planen, gibt es ein Sprint Planning. Nachdem ein Sprint fertig ist, trifft sich das Team und erört in einer Retrospektive, welche Aspekte gut und schlecht gelaufen sind. Die schlechten Aspekte versucht man im nächsten Sprint zu verbessern (vgl. ~\cite{asana_kanban_nodate}).
\\
Im agilen Projektmanagement werden dabei weitere Rahmenkonzepte wie \enquote{Scrum} oder \enquote{Kanban} verwendet. Auf die Konzepte des \enquote{Scrums} wird im Kapitel \ref{sec:solutions_for_habitica} näher eingegangen.
\subsubsection{Kanban}
Kanban ist ein Teilbereich des Agile-Methode und erweitert diesen Ansatz. Die Herangehensweise ist dabei, Teams dabei zu unterstützen, ihre Arbeit in Übereinstimmung mit den individuellen Kapazitäten jedes Teammitgliedes zu bringen und dabei Engpässe zu vermeiden. Die Methode ist darauf ausgerichtet, kontinuierlichen Fortschritt zu fördern, wobei Arbeitsschritte jederzeit aus dem Backlog in den laufenden Arbeitsablauf integriert werden können.
Die aktuell zu bearbeitenden Aufgaben, auch als \enquote{Work in Progress} oder kurz \enquote{Wip} bezeichnet, sind durch ein Wip-Limit beschränkt. Dadurch werden die Durchlaufseiten der Aufgaben verkürzt und das Team kann agiler und flexibler auf neue Gegebenheiten reagieren.~\cite{asana_kanban_13_4_23}
\\
Der Kanban-Ansatz wird mithilfe eines Kanban Boards implementiert. Es handelt sich dabei um eine Form des visuellen Projektmanagements, mit dem Teams ihre Arbeitslast und Abläufe besser visualisieren können. Ein Kanban-Board ist eine aus verschiedenen Spalten bestehende Tafel. Üblicherweise ist jede Spalte eine Arbeitsphase, in welcher sich die verschiedenen Aufgaben befinden. In der Regel besteht das Kanban-Board aus den Spalten \enquote{ToDo}, \enquote{In Progress} und \enquote{Done}.
Wie das Agile-Manifesto, besteht Kanban ebenfalls aus Werten und Grundprinzipien.
Diese Werte bilden die Grundlage für die Einführung von Kanban: Respekt vor allen Meinungen und Ansichten; Vereinbarung für Verbesserung und das gemeinsame Ziel; alle Fähigkeiten und Ansichten im Team müssen in Balance sein; Transparenz in der Aufgabendarstellung, -wahrnehmung und -erledigung; Verständnis, dass Arbeit ein Fluss von Aktivitäten ist; Kollaboration: gemeinsames Arbeiten steht im Fokus und Nutzendenfokus: die Arbeit fließt immer zu den Nutzenden.
Die 4 Prinzipien lauten:
\begin{itemize}
    \item  Beginne mit dem, was du gerade tust!
    \item Schrittweise, evolutionäre Veränderungen verfolgen
    \item Aktuelle Prozesse, Rollen und Verantwortlichkeiten berücksichtigen!
    \item Fördere Führung auf allen Ebenen der Organisation!
\end{itemize}
Daraus ergeben sich folgende Praktiken: Visualisiere – Das Kanban-Board; Limitiere die parallele Arbeit – WIP-Limit; Manage den Arbeitsfluss – Serviceklassen; Formuliere Prozessregeln; Kontinuierliche Verbesserung – Rückkopplungsschleifen.(vgl. ~\cite{noauthor_organisationshandbuch_nodate})

%\subsubsection{Projektmanagement Tools}
%Projektmanagement Tools dienen in der Regel als eine Art digitales Kanban Board mit dem Teams ihre Aufgaben sammeln und organisieren können

\subsection{Habitica}
Habitica ist ein Computerspiel, welches das Leben des Anwenders \enquote{gamifiziert}. Es soll ihm dabei helfen, seine Gewohnheiten im realen Leben zu verbessern. Um das zu erreichen, wandelt es alle Aufgaben des Anwenders, dazu zählen Gewohnheiten, Tagesaufgaben und grundlegende To-Dos, in kleine Monster, die besiegt werden müssen. Der Anwender schreitet dabei im Spiel voran, je mehr To-Dos und Tagesaufgaben erledigt werden, oder er seine Gewohnheiten verbessert. Vernachlässigt er seine Tagesaufgaben oder hält sich nicht an seine Gewohnheiten, so fällt er im Spiel zurück.
\\
Für das Abschließen von Aufgaben erhält der Anwender Erfahrungspunkte, Mana und Gold. Durch die Erfahrungspunkte \enquote{levelt} der Avatar des Anwenders hoch. Dadurch kann er dem Avatar verschiedene Attributspunkte verteilen, wodurch der Avatar besser wird. Das Mana benötigt der Anwender um bestimmte Fähigkeiten, die der Avatar besitzt auszuführen. Durch die Goldmünzen kann sich der Anwender Ausrüstung und Quests kaufen. Zusätzlich zu den Spielgegenständen, kann sich der Anwender ein Haustier zur Seite stellen, welches aus einem Ei schlüpft und mit Elixieren und Futter ernährt werden muss.
\\
\\
Es ist möglich Habitica alleine zu spielen oder im Gruppenverbund. Im Kontext des Spiels wird hierbei von einem Party gesprochen. In der Party können nun gemeinsam Quests erledigt werden. Dabei werden bspw. durch das Abschließen von Tagesaufgaben oder To-Dos Gegenstände gesammelt oder verschiedene Monster angegriffen. Einige Quests müssen mit mindestens einer anderen Person bestritten werden, manche können jedoch auch im Einzelspieler Modus, ohne Gruppe, erledigt werden. Außerdem ist es möglich, sich gegenseitig in der Party Herausforderungen zu stellen, um herauszufinden wer der bessere Spieler ist. Zusätzlich bietet die Party eine Chatmöglichkeit mit-sich, durch welchen die Party-Mitglieder miteinander schreiben können.
\\
Bevorzugte Anwendungsgebiete des Spiels und seines Gamifikation-Ansatzes sind dabei sowohl die Ausbildung der Anwender, als auch die Gesundheitliche Förderung und Erholung selbiger. Dabei sollen gute Noten und verbessertes Verhalten, sowie Aufgaben, die das Wohlbefinden verbessern belohnt werden.~\cite{noauthor_funktionen_nodate}
\\
Habitica implementiert ein abgewandeltes Kanban Board, in welchem der Anwender seine Gewohnheiten, Tagesaufgaben und To-Dos anlegen kann. Jedes To-Do und Tagesaufgabe können mit einem Dateum versehen werden, wie es bei Kanban Boards üblich ist. 
\section{Ansatz der Konzeption}

\subsection{Analyse von Habitica}
Dieser Abschnitt beschäftigt sich mit einer Nutzer- und Psychologischen- Analyse, die an der Ludwig-Maximilians Universität in München im Jahr 2018 durchgeführt wurde.~\cite{diefenbach_counterproductive_2019} Das referenzierte Paper untersucht \enquote{kontraproduktive Effekte der Gamifizierung}, welche entstehen, wenn Gamifizierungselemente nicht das beabsichtige Verhalten fördern, sondern das Gegenteil (z.B. das Aufschieben von Tasks, statt sie zu Erledigen). 
Die angesprochenen \enquote{kontraproduktiven Nebeneffekte} können in 2 Kategorien eingeteilt werden. Zum einen können die Gamifikationelemente eine Kontraproduktive Wirkung auf die Motivation von negativen Verhalten auslösen und zum anderen können sie eine Demotivation von positiven Verhalten auslösen (vgl. Tabelle 1 \cite{diefenbach_counterproductive_2019}).
\\
Die Ergebnisse der 2 Studien, die in dem Paper durchgeführt wurden, werden im Folgenden zusammengefasst.
\\
\subsubsection{Studie 1: Interpretative Phenomenological Analysis}
In der erste Studie, der  Interpretative Phenomenological Analysis (IPA)~\cite{smith_reflecting_2004}, wurden Haupt kontraproduktiven Effekte definiert, die der Proband in der Studie erlebt hat.

\paragraph{Bestrafung für Produktivität}\label{sec:cpe1}
In Phasen erhöhter Produktivität, in denen zahlreiche Aufgaben erledigt werden, bleibt möglicherweise wenig Zeit, die entsprechenden Aufgaben in Habitica aufzuhaken. Das führt dazu, dass das Spiel den Anwender bestraft, obwohl der Anwender in der realen Welt produktiv war.\\
$\rightarrow$ Demotivation positiven Verhaltens

\paragraph{Bestrafung für das Versagen  anderer}\label{sec:cpe2}
Bei der Beteiligung an einer Quest mit anderen Benutzern kann man bestraft werden, selbst wenn man alle eigenen täglichen Aufgaben erfüllt und abgehakt hat, aufgrund der Nachlässigkeit anderer.\\
$\rightarrow$ Demotivation positiven Verhaltens

\paragraph{Belohnung für Prokrastination}\label{sec:cpe3}
Die Belohnungsstruktur des Systems ermutigt Anwender dazu, Aufgaben spontan auf den nächsten Tag zu verschieben. Das kann getan werden um nicht vollständig abgeschlossene Aufgaben aufzuheben und später die Belohnung dafür zu erhalten. In der Praxis wird dadurch die Prokrastination unterstützt, anstatt den Task direkt zu erledigen.\\
$\rightarrow$ Motivation von negativem Verhalten

\paragraph{Belohnung für irrelevante Aufgaben}\label{sec:cpe4}
Es ist möglich Aufgaben zu definieren, für die man nichts machen muss und für das Abschließen dieser Aufgaben von Habitica belohnt wird. Allerdings erfordert dies einen erheblichen Zeitaufwand und lenkt die Aufmerksamkeit von den wesentlichen Aufgaben ab, die eigentlich priorisiert werden sollten.
$\rightarrow$ Motivation von negativem Verhalten

\paragraph{Cheating}\label{sec:cpe5}
Auch wenn der Proband seine Aufgaben nicht erfüllte, ermöglicht das System eine Funktion, um trotzdem Belohnungen zu erhalten und Strafen rückgängig zu machen (Charakterstatistiken wiederherstellen).
$\rightarrow$ Indirekte Demotivation von positivem Verhalten

\paragraph{Risikofreie Belohnungen 1}\label{sec:cpe6}
In Habitica kann man Aufgaben so gestalten, dass sie Belohnungen erhalten können, ohne das Risiko einer Bestrafung einzugehen.
$\rightarrow$ Indirekte Demotivation von positivem Verhalten

\paragraph{Risikofreie Belohnungen 2}\label{sec:cpe7}
Habitica ermöglicht es, Aufgaben als positive Gewohnheiten umzubenennen und somit das Risiko von Bestrafungen zu verhindern.
$\rightarrow$ Indirekte Demotivation von positivem Verhalten 
\\
\\
\subsubsection{Meta-Wahrnehmungen}
Zusätzlich zu den Kategorisierten kontraproduktiven Effekte, wurden \enquote{Meta-Wahrnehmungen} der kontraproduktiven Effekte des Belohnungs-/ Bestrafungssystems in Habitica definiert. Dabei handelt es sich um Aussagen aus der ersten Studie, die nicht nur auf spezifische Effekte eingehen, sondern generelle Bedenken hinsichtlich der Unangemessenheit dieses Systems reflektieren. Diese \enquote{Meta-Wahrnehmungen} erfassen auf einer übergeordneten Ebene die grundlegenden Probleme von Habiticas Belohnungs- und Bestrafungssystem und können dabei kontraproduktive Effekte beeinflussen.

\paragraph{Belohnungen sind zu gering}\label{sec:acpe1}
Die Belohnungen in Habitica sind oft zu klein und entsprechen nicht den investierten Aufwand wieder. Zusätzlich sind die Belohnung für eine sinnvolle Investition zu gering. Man muss teilweise viele Aufgaben erledigen und viel Sparen um sich neue Gegenstände oder Quests kaufen zu können.

\paragraph{Belohnungen sind bedeutungslos}\label{sec:acpe2}
Erhaltene Belohnungen erscheinen oft als sinnlos. Rein quantitative Belohnungen wie Punkte könnten auf Dauer demotivierend sein. Es könnte dazu kommen, dass das Zählen der Punkte die Motivation der Anwender untergraben könnten. Qualitative Belohnungen wie z.B. neue Accessoires im Spiel könnten für ihn im Allgemeinen attraktiver sein, allerdings ist die Präsentation im Spiel so dargestellt, dass daraus keine Bedeutung abgeleitet werden kann.

\paragraph{Bestrafungen sind zu hart und unfair}\label{sec:acpe3}
Bestrafungen sind oft praktisch nicht ausgleichbar. Zum Beispiel verliert man Lebenspunkte, die nur dann behalten werden können, wenn man durch die Erfahrungspunkte das nächste Level erreicht. Dies gestaltet sich jedoch schwierig, da das Spiel progressiv ist und mit jedem aufgestiegenem Level schwieriger wird. Um dies zu verhindern, ist die einzige Möglichkeit ein teures Elixier zu kaufen, das so viel kostet wie das Erfüllen von hunderter Aufgaben. Durch das Erleben dieser Ungerechtigkeit durch die Bestrafungen führt dazu, dass Anwender solche Funktionen deaktivieren, bis kaum noch \enquote{gamified} Inhalte in Habitica enthalten sind.

\paragraph{Ungleichgewicht von Belohnungen und Bestrafungen}\label{sec:acpe4}
Der Verlust von Lebenspunkten geschieht schneller als der Erhalt von Belohnungen, was dazu führt, dass die Motivation, auf negative Gewohnheiten zu klicken, abnimmt. Es gibt zahlreiche Situationen, in denen Lebenspunkte ohne ersichtlichen Grund verloren gehen, und die Befürchtung, noch mehr Lebenspunkte zu verlieren , wenn man schlechte Gewohnheiten angibt, verstärkt diese Tendenz. 
\\
\\
\subsubsection{Selbstreflexion}
Im letzten Teil des ersten Tests ging es um die Selbstreflexion der psychologischen Reaktionen auf die kontraproduktiven Effekte von Habitica. Es wurden drei zusätzliche Themen zur Selbstreflexion gezeigt, wie Anwender generell auf die erlebten kontraproduktiven Effekte in Habitica reagieren können.

\paragraph{Enttäuschung}\label{sec:sr1}
Beim Definieren, Erfüllen und abschließenden Abhaken irrelevanter Aufgaben zur Erlangung von Belohnungen (vgl. \ref{sec:cpe4}) blieb keine Zeit und kein empfundener Drang, die tatsächlich wichtigen Aufgaben zu erledigen. Habitica erfüllte dabei nicht die unterstützende Rolle, die sich Anwender wünschen würden. Die Vorstellung, dass das System dem Proband dabei helfen könnte, eine angestrebte Veränderung im Leben vollständig umzusetzen, schwand gänzlich. Stattdessen kann man das Gefühl haben, dass das System einen in die falsche Richtung lenken könnte.

\paragraph{Gefühl, nicht ernst genommen zu werden}\label{sec:sr2}
Der Proband wurde vom System enttäuscht. In ihm äußerte sich das Gefühl, dass das System sein Anliegen nicht ernst nimmt. Das geschieht dadurch, dass die Wahrnehmung durch die Anreizstruktur des Systems verstärkt wird, dass der Fokus darauf liegt leichte und eher unwichtigere Aufgaben zu erledigen um im Spiel voran zu kommen. 
Dadurch, dass das System eher starr ist, muss viel Zeit investiert werden um Workarounds, wie das Anpassen der Charakterstatistiken, zu schaffen seine Aufgaben so in wie man sie haben möchte in die mangelhafte Struktur zu integrieren. Dadurch werden auch Teile des Spielerlebnisses beeinträchtigt.

\paragraph{Negative Erwartungen}\label{sec:sr3}
Die zukünftige Nutzung des Tools weckte beim Proband negative Erwartungen. Die Häufigkeit der Nutzung nahm ab und einige Funktionen wurden deaktiviert. Trotz des Wunsches nach einer weiteren Chance für das Tool überwog die Sorge, dass zu viel Zeit für individuelle Anpassungen aufgebracht werden müsste. Dies führte zu pessimistischen Erwartungen bezüglich der langfristigen Nutzung des Tools. Die erlebten Probleme verdeutlichen jedoch die Herausforderungen bei der sinnvollen Integration von spielerischen Elementen in den Alltag. Bedauerlicherweise konnte Habitica keine Umgebung bereitstellen, die für ihn geeignet war, und die erlebten Probleme führten dazu, dass die ursprünglichen Ziele nicht erreicht wurden. Insgesamt wurde der Effekt als in die falsche Richtung weisend wahrgenommen.

\subsubsection{Feldstudie 2: Existenz und Konsequenzen kontraproduktiver Effekte}
Die zweite Studie untersuchte das Vorhandensein und die Folgen von kontraproduktiven Gamification-Effekten in Habitica anhand einer Stichprobe von 45 Benutzern über  einen Zeitraum von zwei Wochen.

Es wird dabei ein direkter Bezug auf die Ergebnisse von Studie 1 bezogen. Die Zielsetzung der Feldstudie fokussierte sich auf drei Forschungsfrageblöcke, bestehend aus explorativen Fragen (Q) und überprüfbaren Hypothesen (H).

Nachdem in Studie 1 spezifische kontraproduktive Effekte in Habitica identifiziert wurden, richtet sich das Interesse der Feldstudie auf die Ausdehnung dieser Erkenntnisse auf andere Habitica Nutzer. Ziel war es, die allgemeine Verbreitung dieser Effekte zu untersuchen und potenzielle zusätzliche kontraproduktive Effekte zu identifizieren, die von anderen Nutzern berichtet wurden. Konkret wurden die folgenden Fragen und Hypothesen erforscht:

\begin{itemize}
    \item  Erleben Anwender dieselben kontraproduktiven Effekte wie in Studie 1? (Q1)
    \item Welche Effekte sind am häufigsten und schwerwiegendsten? (Q2)
    \item Wie reflektieren Benutzer auf potenziell verwandte Spielmechanismen wie das Belohnungs-/ Bestrafungssystem auf der Meta-Wahrnehmung? Sehen Anwender in Studie 2 Habiticas Belohnungs-/ Bestrafungssystem genauso problematisch wie in Studie 1? (Q3)
\end{itemize}

Wie bereits erläutert und gestützt auf den Erzählungen des Probanden aus Studie 1 wurde angenommen, dass es eine Verbindung zwischen dem direkten Erleben spezifischer kontraproduktiver Effekte und der übergeordneten Wahrnehmungen der Angemessenheit des Belohnungs-/ Bestrafungssystems in Habitica gibt. Infolgedessen wurde prognostiziert, dass Anwender, die intensive Erfahrungen mit kontraproduktiven Effekten gemacht haben, wahrscheinlich das Belohnungs-/ Bestrafungssystem als weniger angemessen beurteilen würden. Daraus ergaben sich folgende Thesen:

\begin{itemize}
    \item  Die Erfahrung kontraproduktiver Effekte korreliert mit Meta-Wahrnehmungen kontraproduktiver Effekte. je mehr kontraproduktive Effekte Anwender erleben, desto weniger angemessen werden sie das Belohnungssystem bewerten (H1a) und das Bestrafungssystem (H1b).
\end{itemize}

Zum Schluss wurde überprüft, ob Probanden zusätzliche Effekte aufdecken konnten, die über die Studie 1 hinausgingen:

\begin{itemize}
    \item  Werden die Probandenberichte in Studie 2 zusätzliche kontraproduktive Effekte aufdecken? Wenn ja, können Ähnlichkeiten oder Cluster in Benutzerberichten zu zusätzlichen kontraproduktiven Effekten identifiziert werden? (Q4)
\end{itemize}

\subsection{Analyse von Asana}

\subsection{Lösungsansätze für Habitica}\label{sec:solutions_for_habitica}

\section{Fazit und Ausblick}

%Some examples of references. A paginated journal article~\cite{Abril07}, an enumerated journal article~\cite{Cohen07}, a reference to an entire issue~\cite{JCohen96}, a monograph (whole book) ~\cite{Kosiur01}, a monograph/whole book in a series (see 2a in spec. document)~\cite{Harel79}, a divisible-book such as an anthology or compilation~\cite{Editor00} followed by the same example, however we only output the series if the volume number is given~\cite{Editor00a} (so Editor00a's series should NOT be present since it has no vol. no.), a chapter in a divisible book~\cite{Spector90}, a chapter in a divisible book in a series~\cite{Douglass98}, a multi-volume work as book~\cite{Knuth97}, an article in a proceedings (of a conference, symposium, workshop for example) (paginated proceedings article)~\cite{Andler79}, a proceedings article with all possible elements~\cite{Smith10}, an example of an enumerated proceedings article~\cite{VanGundy07}, an informally published work~\cite{Harel78}, a doctoral dissertation~\cite{Clarkson85}, a master's thesis~\cite{anisi03}, an finally two online documents or world wide web resources~\cite{Thornburg01, Ablamowicz07}.

\bibliographystyle{ACM-Reference-Format}
\bibliography{sample}

\end{document}
\endinput

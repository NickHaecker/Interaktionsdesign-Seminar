% Holodecks template version of 2023-09-01 enhances the ACM template, version 1.7.0:
% https://www.acm.org/publications/proceedings-template
% The ACM Latex guide provides further information about the ACM template

\documentclass[sigconf, nonacm]{acmart}
\usepackage[german]{babel}
\usepackage{csquotes}
%% The following content must be adapted for the final version
% paper-specific
\newcommand\holodoi{XX.XX/XXX.XX}
\newcommand\holopages{XXX-XXX}
% issue-specific
\newcommand\holovolume{1}
\newcommand\holoissue{1}
\newcommand\holoyear{2023}
% should be fine as it is
\newcommand\holoauthors{\authors}
\newcommand\holotitle{\shorttitle} 
% leave empty if no availability url should be set
\newcommand\holoavailabilityurl{URL_TO_YOUR_ARTIFACTS}
% whether page numbers should be shown or not, use 'plain' for review versions, 'empty' for camera ready
\newcommand\holopagestyle{plain} 

\begin{document}
\title{Konzeption eines Projektmanagement-Task-Boards unter Anwendung des Habitica-Gamification-Ansatzes}

%%
%% The "author" command and its associated commands are used to define the authors and their affiliations.
\author{Nick Philipp Häcker}
\affiliation{%
  \institution{Fakultät für Digitale Medien}
  \institution{Hochschule Furtwangen University}
}
\email{nick.athaeck@gmail.com}




%%
%% The abstract is a short summary of the work to be presented in the
%% article.
\begin{abstract}
In dieser Arbeit wird ein Forschungspaper der Ludwig-Maximilians-Universität, München, Fakultät der Psychologie mit dem Namen \enquote{Counterproductive effetcs of gamification: An analysis \allowbreak on the example of the gamified task manager} analysiert. Das Paper analysiert Nebeneffekte, die bei der Nutzung von Habitica auftreten und fasst diese zusammen.

Im weiteren Verlauf wird das Projektmanagement-Werkzeug Asana vorgestellt. Zunächst wird deskriptiv erläutert, welche Eigenschaften ein effektives Projektmanagement-Werkzeug haben sollte. Eine detaillierte Liste der Funktionen von Asana wird präsentiert, begleitet von einer Bewertung durch verschiedene Quellen.

Diese Arbeit verbindet anschließend die identifizierten Nebenwirkungen aus der Habitica-Analyse mit den Funktionen von Asana und stellt konzeptionelle Lösungsvorschläge vor. Diese Vorschläge zielen darauf ab, die auftretenden Nebeneffekte einzudämmen oder ganz zu lösen.

Das Fazit reflektiert die Ergebnisse der Analyse und bietet einen Ausblick auf mögliche zukünftige Entwicklungen im Bereich der gamifizierten Produktivitätswerkzeuge und Projektmanagement-Plattformen. Die Erkenntnisse der vorliegenden Arbeit sollten zu aktuellen Forschung beitragen, um die Effizienz und Effektivität von gamifizierten Produktivitätswerkzeugen und Projektmanagement-Plattformen zu verbessern.
\end{abstract}

\maketitle

%%% do not modify the following Holodeck block %%
%%% Holodeck block start %%%
\pagestyle{\holopagestyle}
\begingroup\small\noindent\raggedright\textbf{CCS Concepts:}\\
CCS $\rightarrow$ Human-centered computing $\rightarrow$ Human computer interaction (HCI) $\rightarrow$ HCI design and evaluation methods $\rightarrow$ User studies;
CCS $\rightarrow$ Theory of computation $\rightarrow$ Theory and algorithms for application domains $\rightarrow$ Algorithmic game theory and mechanism design $\rightarrow$ Network games; CCS $\rightarrow$ Software and its engineering $\rightarrow$ Software organization and properties $\rightarrow$ Contextual software domains $\rightarrow$ Virtual worlds software $\rightarrow$ Interactive games


 \renewcommand\thefootnote{}\footnote{\noindent
	Seminar Interaktionsdesign \\
	\emph{Interaktionsdesign MIM1}, WS 2023/2024\\
	Dozent: Prof. Dr. Thomas Schlegel\\
	Modul: Interaktionsdesign, Fakultät Digitale Medien
 \\
 }\addtocounter{footnote}{-1}\endgroup
%%% Holodecks block end %%%

%%% do not modify the following Holodecks block %%
%%% Holodecks block start %%%
\ifdefempty{\holoavailabilityurl}{}{
\vspace{.3cm}
\begingroup\small\noindent\raggedright\textbf{Keywords}\\
Gamification, Asana, Task-Management
% The source code, data, and/or other artifacts have been made available at \url{\holoavailabilityurl}.
\endgroup
}
%%% Holodecks block end %%%

\section{Einleitung}
In der heutigen Welt stehen sowohl viele Studierende als auch Berufstätige vor einer Vielzahl von Herausforderungen, um ihre Produktivität aufrechtzuerhalten. Um das zu erreichen gibt es viele verschiedene Möglichkeiten. Dabei werden Aufgaben häufig in Papier Format aufgeschrieben, oder es werden - abhängig von der Branche - Tools wie Notion, GitHub Projects oder Habitica verwendet.
Habitica ist ein \enquote{Habbit-Tracker}-Tool welches im weiteren Verlauf des Papers im Detail vorgestellt wird. Bei der Verwendung der Anwendung treten für den Anwender unerwünschte Nebeneffekte auf, welche bei einer Forschungsstudie der Ludwig-Maximilians-Universität aufgedeckt wurden. \\
In dieser Arbeit wird der Versuch unternommen Konzepte und Implementierungen aus Projektmanagement-Tools wie Asana mit dem bestehenden \enquote{Habbit-Tracker}-Tool Habitica zu vereinen. Hierdurch wird erwartet, dass bestehende Nebeneffekte reduziert oder gar aufgelöst werden. Zudem werden die Vorteile der beiden Anwendungen präsentiert. Auf Basis dieser Vorteile wird der Versuch unternommen, eine weiterentwickelte Ausführung der Kombination der beiden Tools herauszuarbeiten.
\\
\\
Die Forschungsfrage dieses Paper lautet daher:\\
\textbf{Wie können paradoxe Nebeneffekte der Bedienung \allowbreak von Habitica durch das Miteinbeziehen eines Projektmanagement Tools wie Asana vermindert werden?}
\\
\\
In Kapitel \ref{sec:pmtia} (\nameref{sec:pmtia}) wird zunächst auf die Grundlagen des Projektmanagements und dessen Konzepte eingegangen. Im darauffolgenden Kapitel \ref{sec:habitica} wird der \enquote{Habbit-Tracker} Habitica im Allgemeinen vorgestellt. In Kapitel \ref{sec:conception} (\nameref{sec:conception}) wird das Konzept der Überarbeitung vorgestellt. Zur Grunde liegt dafür zunächst die Analyse des referenzierten Papers in Kapitel \ref{sec:analysis}. Anschließend erfolgt das Vorstellen von Asana (Kapitel \ref{sec:asana}). Darauffolgend werden die Lösungsvorschläge zusammenfassend erläutert (Kapitel \ref{sec:solutions_for_habitica}). Abschließend erfolgt das Fazit und ein Ausblick auf weiterführende Forschung-/ Entwicklungen (Kaitel \ref{sec:fazit}).


\subsection{Projektmanagement und Tools im Allgemeinen}\label{sec:pmtia}
Das Wort Projektmanagement besteht aus den zwei Wörtern \enquote{Projekt} und \enquote{Management}. 
In diesem Kontext wird ein Projekt nach der DIN 69901-5 als eine \enquote{Absicht, die im Wesentlichen durch die Einzigartigkeit der Bedienung in ihrer Gesamtheit gekennzeichnet ist} Es zeichnet sich durch eine Zielvorgabe mit zeitlichen, finanziellen und personellen Einschränkungen aus.~\cite{DIN69901-5} 
Der Begriff \enquote{Management} \enquote{bezeichnet heute im betriebswirtschaftlichen Sprachgebrauch einerseits - in funktionaler Perspektive - die Tätigkeit der Unternehmensführung. Andererseits wird auch - in institutioneller Perspektive - das geschäftsführende Organ, also die Gruppe der leitenden Personen eines Unternehmens als Management bezeichnet.}~\cite{haric_definition_nodate} 
Die Managementaufgaben umfassen die Definition von Organisationszielen, die Ausarbeitung einer Strategie zur Zielerreichung sowie die Organisation und Koordination der Produktionsfaktoren als auch die Führung der Mitarbeiter.
Kombiniert bezeichnet der Begriff \enquote{Projektmanagement} die \enquote{Gesamtheit von Führungsaufgaben, -organisation, -techniken und -mitteln für die Initiierung, Definition, Planung, Steuerung und den Abschluss von Projekten}~\cite{DIN69901-5}
\\
Generell hilft das Projektmanagement einem Team bei der Organisation, Nachtverfolgung und Durchführung bei der Projektarbeit. Innerhalb des Projektmanagements gibt es verschiedene Arten, Methoden und Ansätze. Dazu zählen bspw. das Agile Projektmanagement, das Wasserfallmodell, die PRINCE2-Methode und die Methode des kritischen Pfades.~\cite{asana_projektmanagement_nodate} Um die Forschungsfrage zu beantworten, den Mechanismus und Ablauf des Projektmanagements mit der Anwendung Habitica zu kombinieren, wird im Folgenden nur auf das Agile Projektmanagement eingegangen.

\subsubsection{Agiles Projektmanagement}
Das \enquote{Agile Projektmanagement beschreibt eine interaktive Methodik, bei der Arbeitsphasen in kurze Sprint aufgeteilt werden.}~\cite{asana_kanban_nodate} Für die Anwender ist dieser Ansatz am Flexibelsten, da sie in bestehenden kurzen Zeitabschnitten Zwischenergebnisse präsentieren können, und schnell auf Änderungen der Gegebenheiten reagieren können und diese einplanen. Allerdings kann dadurch ungeplante Mehrarbeit entstehen.
Das Agile Projektmanagement basiert auf iterativen Prozessen, wie dem Backlog-Management, Sprints, Retrospektiven, Iterationen und weiteren Sprints.
Das Backlog ist eine Liste von Aufgaben, die während des Sprints bearbeitet werden können. Diese Aufgaben stehen in der Regel in einem Kanban Board, welches im folgenden Kapitel vorgestellt wird. Sprints dienen einem Zeitrahmen, in welchem eine bestimmte Anzahl von Aufgaben aus dem Backlog erledigt werden sollen. Um diese Aufgaben, die erledigt werden sollen zu planen, gibt es ein Sprint Planning. Nachdem ein Sprint fertig ist, trifft sich das Team und erört in einer Retrospektive, welche Aspekte gut und schlecht gelaufen sind. Die schlechten Aspekte versucht man im nächsten Sprint zu verbessern (vgl. ~\cite{asana_kanban_nodate}).
\\
Im agilen Projektmanagement werden dabei weitere Rahmenkonzepte wie \enquote{Scrum} oder \enquote{Kanban} verwendet. Auf die Konzepte des \enquote{Scrums} wird im Kapitel \ref{sec:solutions_for_habitica} näher eingegangen.

\subsubsection{Kanban}
Kanban ist ein Teilbereich der Agilen Methoden und erweitert diesen Ansatz. Die Herangehensweise besteht darin, Teams dabei zu unterstützen, ihre Arbeit in Übereinstimmung mit den individuellen Kapazitäten jedes Teammitgliedes zu bringen und dabei Engpässe zu vermeiden. Die Methode ist darauf ausgerichtet, kontinuierlichen Fortschritt zu fördern, wobei Arbeitsschritte jederzeit aus dem Backlog in den laufenden Arbeitsablauf integriert werden können.
Die aktuell zu bearbeitenden Aufgaben, auch als \enquote{Work in Progress} oder kurz \enquote{Wip} bezeichnet, sind durch ein Wip-Limit beschränkt. Dadurch werden die Durchlaufseiten der Aufgaben verkürzt und das Team kann agiler und flexibler auf neue Gegebenheiten reagieren.~\cite{asana_kanban_13_4_23}
\\
Der Kanban-Ansatz wird mithilfe eines Kanban Boards implementiert. Es handelt sich dabei um eine Form des visuellen Projektmanagements, mit dem Teams ihre Arbeitslast und Abläufe besser visualisieren können. Ein Kanban-Board ist eine aus verschiedenen Spalten bestehende Tafel. Üblicherweise ist jede Spalte eine Arbeitsphase, in welcher sich die verschiedenen Aufgaben befinden. In der Regel besteht das Kanban-Board aus den Spalten \enquote{ToDo}, \enquote{In Progress} und \enquote{Done}.
Wie das Agile-Manifesto, besteht Kanban ebenfalls aus Werten und Grundprinzipien.
Diese Werte bilden die Grundlage für die Einführung von Kanban: Respekt vor allen Meinungen und Ansichten; Vereinbarung für Verbesserung und das gemeinsame Ziel; alle Fähigkeiten und Ansichten im Team müssen in Balance sein; Transparenz in der Aufgabendarstellung, -wahrnehmung und -erledigung; Verständnis, dass Arbeit ein Fluss von Aktivitäten ist; Kollaboration: gemeinsames Arbeiten steht im Fokus und Nutzendenfokus: die Arbeit fließt immer zu den Nutzenden.
Die 4 Prinzipien lauten:
\begin{itemize}
    \item  Beginne mit dem, was du gerade tust!
    \item Schrittweise, evolutionäre Veränderungen verfolgen
    \item Aktuelle Prozesse, Rollen und Verantwortlichkeiten berücksichtigen!
    \item Fördere Führung auf allen Ebenen der Organisation!
\end{itemize}
Daraus ergeben sich folgende Praktiken (vgl. ~\cite{noauthor_organisationshandbuch_nodate}):
\begin{itemize}
\item Visualisiere – Das Kanban-Board
\item Limitiere die parallele Arbeit – WIP-Limit
\item Manage den Arbeitsfluss – Serviceklassen
\item Formuliere Prozessregeln
\item Kontinuierliche Verbesserung – Rückkopplungsschleifen
\end{itemize}

\subsection{Habitica}\label{sec:habitica}
Habitica ist ein Computerspiel, welches das Leben des Anwenders \enquote{gamifiziert}. Es soll ihm dabei helfen, seine Gewohnheiten im realen Leben zu verbessern. Um das zu erreichen, wandelt es alle Aufgaben des Anwenders in kleine Monster, die besiegt werden müssen, um. Zu den Aufgaben zählen beispielsweise Gewohnheiten, Tagesaufgaben sowie grundlegende To-Dos. Der Anwender schreitet dabei im Spiel voran, je mehr To-Dos und Tagesaufgaben erledigt werden, oder er seine Gewohnheiten verbessert. Vernachlässigt er seine Tagesaufgaben oder hält sich nicht an seine Gewohnheiten, so fällt er im Spiel zurück.
\\
Für das Abschließen von Aufgaben erhält der Anwender Erfahrungspunkte, Mana und Gold. Durch die Erfahrungspunkte \enquote{levelt} der Avatar des Anwenders hoch. Dadurch kann er dem Avatar verschiedene Attributspunkte verteilen, wodurch der Avatar besser wird. Das Mana benötigt der Anwender, um bestimmte Fähigkeiten, die der Avatar besitzt, auszuführen. Durch die Goldmünzen kann sich der Anwender Ausrüstung und Quests kaufen. Zusätzlich zu den Spielgegenständen, kann sich der Anwender ein Haustier zur Seite stellen, welches aus einem Ei schlüpft und mit Elixieren und Futter ernährt werden muss.
\\
\\
Es ist möglich Habitica alleine zu spielen oder im Gruppenverbund. Im Kontext des Spiels wird hierbei von einem Party gesprochen. In der Party können nun gemeinsam Quests erledigt werden. Dabei werden bspw. durch das Abschließen von Tagesaufgaben oder To-Dos Gegenstände gesammelt oder verschiedene Monster angegriffen. Einige Quests müssen mit mindestens einer anderen Person bestritten werden, manche können jedoch auch im Einzelspieler Modus, ohne Gruppe, erledigt werden. Außerdem ist es möglich, sich gegenseitig in der Party Herausforderungen zu stellen, um herauszufinden, wer der bessere Spieler ist. Zusätzlich bietet die Party eine Chatmöglichkeit, durch welchen sich die Party-Mitglieder austauschen können.
\\
Bevorzugte Anwendungsgebiete des Spiels und seines Gamifikation-Ansatzes sind dabei sowohl die Ausbildung der Anwender, als auch die gesundheitliche Förderung und Erholung selbiger. Dabei sollen gute Noten und verbessertes Verhalten sowie Aufgaben, die das Wohlbefinden verbessern, belohnt werden.~\cite{noauthor_funktionen_nodate}
\\
Habitica implementiert ein abgewandeltes Kanban Board, in welchem der Anwender seine Gewohnheiten, Tagesaufgaben und To-Dos anlegen kann. Jedes To-Do und Tagesaufgabe können mit einem Datum versehen werden, wie es bei Kanban Boards üblich ist. \\
Im Folgenden Kapitel wird nun die Konzeption einer neuen Anwendung vorgestellt.

\section{Ansatz der Konzeption}\label{sec:conception}

\subsection{Analyse von Habitica}\label{sec:analysis}
Dieser Abschnitt beschäftigt sich mit einer Nutzer- und Psychologischen- Analyse, die an der Ludwig-Maximilians Universität in München im Jahr 2018 durchgeführt wurde.~\cite{diefenbach_counterproductive_2019} Das referenzierte Paper untersucht \enquote{kontraproduktive Effekte der Gamifizierung}, welche entstehen, wenn Gamifizierungselemente nicht das beabsichtige Verhalten fördern, sondern das Gegenteil (z.B. das Aufschieben von Tasks, statt sie zu Erledigen). 
Die angesprochenen \enquote{kontraproduktiven Nebeneffekte} können in 2 Kategorien eingeteilt werden. Zum einen können die Gamifikationelemente eine kontraproduktive Wirkung auf die Motivation von negativen Verhalten auslösen und zum anderen können sie eine Demotivation von positiven Verhalten auslösen (vgl. Tabelle 1 \cite{diefenbach_counterproductive_2019}).
\\
Die Ergebnisse der 2 Studien, die in dem o.g. Paper durchgeführt wurden, werden im Folgenden zusammengefasst.
\\

\subsubsection{Studie 1: Interpretative Phenomenological Analysis}
In der erste Studie, der  Interpretative Phenomenological Analysis (IPA)~\cite{smith_reflecting_2004}, wurden die am häufigst vorkommenden kontraproduktiven Effekte definiert, die der Proband in der Studie erlebt hat.

\paragraph{Bestrafung für Produktivität}\label{sec:cpe1}
In Phasen erhöhter Produktivität, in denen zahlreiche Aufgaben erledigt werden, bleibt möglicherweise wenig Zeit, die entsprechenden Aufgaben in Habitica abzuhaken. Das führt dazu, dass das Spiel den Anwender bestraft, obwohl der Anwender in der realen Welt produktiv war.\\
$\rightarrow$ Demotivation positiven Verhaltens

\paragraph{Bestrafung für das Versagen  anderer}\label{sec:cpe2}
Bei der Beteiligung an einer Quest mit anderen Benutzern kann man bestraft werden, selbst wenn man alle eigenen täglichen Aufgaben erfüllt und abgehakt hat, aufgrund der Nachlässigkeit anderer.\\
$\rightarrow$ Demotivation positiven Verhaltens

\paragraph{Belohnung für Prokrastination}\label{sec:cpe3}
Die Belohnungsstruktur des Systems ermutigt Anwender dazu, Aufgaben spontan auf den nächsten Tag zu verschieben. Das kann getan werden, um nicht vollständig abgeschlossene Aufgaben aufzuheben und später die Belohnung dafür zu erhalten. In der Praxis wird dadurch die Prokrastination unterstützt, anstatt den Task direkt zu erledigen.\\
$\rightarrow$ Motivation von negativem Verhalten

\paragraph{Belohnung für irrelevante Aufgaben}\label{sec:cpe4}
Es ist möglich Aufgaben zu definieren, für die man keinerlei Tätigkeiten erledigt haben muss und für das Abschließen dieser Aufgaben von Habitica belohnt wird. Allerdings erfordert dies einen erheblichen Zeitaufwand und lenkt die Aufmerksamkeit von den wesentlichen Aufgaben ab, die eigentlich priorisiert werden sollten.
$\rightarrow$ Motivation von negativem Verhalten

\paragraph{Cheating}\label{sec:cpe5}
Auch wenn der Proband seine Aufgaben nicht erfüllte, ermöglicht das System eine Funktion, um trotzdem Belohnungen zu erhalten und Strafen rückgängig zu machen (Charakterstatistiken wiederherstellen).
$\rightarrow$ Indirekte Demotivation von positivem Verhalten

\paragraph{Risikofreie Belohnungen 1}\label{sec:cpe6}
In Habitica kann man Aufgaben so gestalten, dass sie Belohnungen erhalten können, ohne das Risiko einer Bestrafung einzugehen.
$\rightarrow$ Indirekte Demotivation von positivem Verhalten

\paragraph{Risikofreie Belohnungen 2}\label{sec:cpe7}
Habitica ermöglicht es, Aufgaben als positive Gewohnheiten umzubenennen und somit das Risiko von Bestrafungen zu verhindern.
$\rightarrow$ Indirekte Demotivation von positivem Verhalten 
\\
\\
\subsubsection{Meta-Wahrnehmungen}
Zusätzlich zu den kategorisierten kontraproduktiven Effekte, wurden \enquote{Meta-Wahrnehmungen} der kontraproduktiven Effekte des Belohnungs-/ Bestrafungssystems in Habitica definiert. Dabei handelt es sich um Aussagen aus der ersten Studie, die nicht nur auf spezifische Effekte eingehen, sondern generelle Bedenken hinsichtlich der Unangemessenheit dieses Systems reflektieren. Diese \enquote{Meta-Wahrnehmungen} erfassen auf einer übergeordneten Ebene die grundlegenden Probleme von Habiticas Belohnungs- und Bestrafungssystem und können dabei kontraproduktive Effekte beeinflussen.

\paragraph{Belohnungen sind zu gering}\label{sec:acpe1}
Die Belohnungen in Habitica sind oft zu klein und entsprechen nicht den investierten Aufwand wieder. Zusätzlich sind die Belohnung für eine sinnvolle Investition zu gering. Man muss teilweise viele Aufgaben erledigen und viele Münzen sparen um sich neue Gegenstände oder Quests kaufen zu können.

\paragraph{Belohnungen sind bedeutungslos}\label{sec:acpe2}
Erhaltene Belohnungen erscheinen oft als sinnlos. Rein quantitative Belohnungen, wie Punkte könnten auf Dauer demotivierend sein. Es könnte dazu kommen, dass das Zählen der Punkte die Motivation der Anwender untergraben könnte. Qualitative Belohnungen, wie z.B. neue Accessoires im Spiel könnten für ihn im Allgemeinen attraktiver sein, allerdings ist die Präsentation im Spiel so dargestellt, dass daraus keine Bedeutung abgeleitet werden kann.

\paragraph{Bestrafungen sind zu hart und unfair}\label{sec:acpe3}
Bestrafungen sind oft praktisch nicht ausgleichbar. Zum Beispiel verliert man Lebenspunkte, die nur dann behalten werden können, wenn man durch die Erfahrungspunkte das nächste Level erreicht. Dies gestaltet sich jedoch schwierig, da das Spiel progressiv ist und mit jedem aufgestiegenem Level schwieriger wird. Um dies zu verhindern, ist die einzige Möglichkeit ein teures Elixier zu kaufen, das so viel kostet wie das Erfüllen von hunderter Aufgaben. Durch das Erleben dieser Ungerechtigkeit der Bestrafungen führt dazu, dass Anwender solche Funktionen deaktivieren, bis kaum noch gamifizierte Inhalte in Habitica enthalten sind.

\paragraph{Ungleichgewicht von Belohnungen und Bestrafungen}\label{sec:acpe4}
Der Verlust von Lebenspunkten geschieht schneller als der Erhalt von Belohnungen, was dazu führt, dass die Motivation, auf negative Gewohnheiten zu klicken, abnimmt. Es gibt viele Situationen, in denen Lebenspunkte unerklärlicherweise verloren gehen. Die Angst, durch das Aufgeben schlechter Gewohnheiten noch mehr Lebenspunkte zu verlieren, verstärkt diese Tendenz nur noch.
\\
\\
\subsubsection{Selbstreflexion}
Im letzten Teil des ersten Tests ging es um die Selbstreflexion der psychologischen Reaktionen auf die kontraproduktiven Effekte von Habitica. Es wurden drei zusätzliche Themen zur Selbstreflexion gezeigt, wie Anwender generell auf die erlebten kontraproduktiven Effekte in Habitica reagieren können.

\paragraph{Enttäuschung}\label{sec:sr1}
Beim Definieren, Erfüllen und abschließenden Abhaken irrelevanter Aufgaben zur Erlangung von Belohnungen (vgl. \ref{sec:cpe4}) blieb keine Zeit und kein empfundener Drang, die tatsächlich wichtigen Aufgaben zu erledigen. Habitica erfüllte dabei nicht die unterstützende Rolle, die sich Anwender wünschen würden. Die Vorstellung, dass das System dem Proband dabei helfen könnte, eine angestrebte Veränderung im Leben vollständig umzusetzen, schwand gänzlich. Stattdessen kann man das Gefühl haben, dass das System einen in die falsche Richtung lenken könnte.

\paragraph{Gefühl, nicht ernst genommen zu werden}\label{sec:sr2}
Der Proband wurde vom System enttäuscht. In ihm äußerte sich das Gefühl, dass das System sein Anliegen nicht ernst nimmt. Das geschieht dadurch, dass die Wahrnehmung durch die Anreizstruktur des Systems verstärkt wird, dass der Fokus darauf liegt, leichte und eher unwichtigere Aufgaben zu erledigen um im Spiel voran zu kommen. 
Durch die mangelnde Flexibilität des Systems muss viel Zeit investiert werden, um passende Workarounds zu schaffen, wie bspw. das Anpassen der Charakterstatistiken oder das Einpflegen von Aufgaben nach Belieben des Nutzers. Dadurch werden auch Teile des Spielerlebnisses beeinträchtigt.

\paragraph{Negative Erwartungen}\label{sec:sr3}
Die Aussicht auf eine zukünftige Anwendung des Tools löste beim Probanden negative Vorahnungen aus. Die Nutzungsfrequenz ging zurück und einige Funktionen wurden ausgeschaltet. Obwohl der Wunsch bestand, dem Tool eine weitere Chance zu geben, war die Befürchtung groß, dass die individuelle Anpassung zu zeitaufwendig sein könnte. Dies resultierte in einer pessimistischen Haltung hinsichtlich der langfristigen Anwendung des Tools. Die während der Nutzung aufgetretenen Probleme unterstreichen jedoch die Schwierigkeiten bei der effektiven Einbindung von spielerischen Elementen in den Alltag. Leider konnte Habitica keine passende Umgebung für den Probanden schaffen, und die aufgetretenen Probleme verhinderten das Erreichen der ursprünglichen Ziele. Insgesamt wurde der Effekt als kontraproduktiv empfunden.

\subsubsection{Feldstudie 2: Existenz und Konsequenzen kontraproduktiver Effekte}
Die zweite Studie untersuchte das Vorhandensein und die Folgen von kontraproduktiven Gamification-Effekten in Habitica anhand einer Stichprobe von 45 Probanden über einen Zeitraum von zwei Wochen.

Es wird dabei ein direkter Bezug auf die Ergebnisse von Studie 1 bezogen.

\paragraph{Prävalenz und Erfahrung mit kontraproduktiven Wirkungen}

Dieser Fragen-\ Hypothesenkomplex zielt darauf ab, die Erfahrungen der Benutzer mit kontraproduktiven Effekten in einem Belohnungs-/Bestrafungssystem zu untersuchen. Sie analysiert, wie häufig und schwerwiegend diese Effekte sind, wie die Benutzer darauf reagieren und ob es zusätzliche kontraproduktive Effekte gibt, die in den Benutzerberichten identifiziert werden können. Die Studie postuliert auch, dass die Erfahrung kontraproduktiver Effekte die Wahrnehmung des Belohnungs- und Bestrafungssystems beeinflusst.

\paragraph{Korrelationen zwischen kontraproduktiven Effekten und Nutzererfahrung}

Dieser Absatz untersucht, wie kontraproduktive Effekte und die Wahrnehmung dieser Effekte die Zufriedenheit und wahrgenommene Nützlichkeit von Habitica beeinflussen. Sie postuliert, dass negative Erfahrungen und Wahrnehmungen zu niedrigeren Bewertungen führen. Darüber hinaus wird die Veränderung der Motivation zur Nutzung von Habitica über einen Zeitraum von zwei Wochen sowie mögliche Korrelationen zwischen kontraproduktiven Effekten und Motivationsänderungen untersucht.

\paragraph{Interindividuelle Unterschiede}

Der letzte Absatz untersucht auch, wie die Überzeugung der Benutzer in Gamification ihre Zufriedenheit, wahrgenommene Nützlichkeit und Nutzungshäufigkeit von Habitica beeinflusst. Darüber hinaus wird geprüft, ob es Korrelationen zwischen dem Erleben oder der Meta-Wahrnehmung von kontraproduktiven Effekten und der individuellen Überzeugung in Gamification oder dem individuellen Benutzertyp gibt.

\paragraph{Methodik}
Im Rahmen der Untersuchung zu den Forschungsfragen wurden verschiedene Messungen durchgeführt, die die Prävalenz und das Erleben von kontraproduktiven Effekten in Habitica abdecken. Zudem wurden prominente Aspekte der Benutzererfahrung, wie Zufriedenheit, Nützlichkeit und Motivation, bewertet. Diese Aspekte sind wichtige Bestandteile von Modellen zur Benutzererfahrung und Technologieakzeptanz.
Darüber hinaus wurden potenziell relevante individuelle Variablen erfasst, darunter die Überzeugung in Gamification und die individuelle Motivation für die Nutzung gamifizierter Systeme. Die Messungen erstreckten sich auch auf die Statistiken zum Nutzungsverhalten, wie die Verwendung von Funktionen in Habitica, die Nutzungsintensität sowie die vorherigen Aufgabenverwaltungsroutinen der Teilnehmer.

\paragraph{Ergebnisse}$~$

\textbf{Prävalenz und Erfahrung mit kontraproduktiven Effekten}
\\
%Sowohl Tabelle 3 als auch Schaubild 3 zeigen, dass alle Teilnehmer in gewissem Maße mindestens einen der abgefragten kontraproduktiven Effekte (CPE) erlebt hat. Kein Teilnehmer gab für alle CPEs in der Bewertung an, dass er diese nicht erlebt hat. Die verschiedenen CPEs unterscheiden sich jedoch in ihrer Häufigkeit des Auftretens und nicht alle Benutzer erlebten es in der Schwere, wie es in Studie 1 der Fall war. \enquote{\nameref{sec:cpe1}}, \enquote{\nameref{sec:cpe4}} und \enquote{\nameref{sec:cpe6}} sind die dabei am häufigsten aufgetretenen kontraproduktiven Effekte. 
%Im Hinblick auf Q1 konnte festgestellt werden, dass alle in Studie 1 beschriebenen CPEs auch von mindestens einigen Teilnehmern in Studie 2 bestätigt wurden. Das bedeutet, dass das Erleben von kontraproduktiven Effekten ein allgemeines Problem ist. Es gibt jedoch erhebliche Unterschiede in der Prävalenz und erlebten Ernsthaftigkeit der verschiedenen CPEs bei Bezug auf Q2. Der am häufigsten erlebte kontraproduktive Effekt ist \enquote{\nameref{sec:cpe4}}. Der am wenigsten erlebte CPE ist \enquote{\nameref{sec:cpe5}}. In Bezug auf Q3 ergibt die Auswertung, dass die Probanden das Belohnungs-/ Bestrafungssystem als eher angemessen betrachten. 
%Wie in H1 angenommen korrelieren die Erfahrungen konkreter kontraproduktiver Effekte und die Meta-Wahrnehmungen von kontraproduktiven Effekten miteinander. Allerdings tun sie das nur in Bezug auf die Erfahrung der Benutzer mit dem Belohnungssystem. Ebenfalls mit H1 korreliert die Angemessenheit von Belohnungen negativ mit dem durchschnittlichen CPE, was darauf hinweist, dass diejenigen Anwender, die das Belohnungssystem als nicht angemessen betrachten, dazu neigen könnten, kontraproduktive Nebeneffekte herbeizurufen. 
%\\
%Im Bezug auf Q4 konnten weitere CPEs identifiziert werden. Zum einen gibt es einige Überschneidungen zu denen, die in Studie 1 gefunden wurden. Zum anderen konnten die restlichen in 3 Cluster geordnet werden. 
%Eins der Cluster bezieht sich auf die fehlende Flexibilität von Habitica, um unterschiedliche Fälligkeitsdaten zuzuweisen oder das Abhaken von Aufgaben nach Mitternacht zu akzeptieren. 
%\\
%Das zweite Cluster bezieht sich auf den subjektiven Verlust beim Erreichen von Zielen. Das bedeutet, dass einige Probanden angegeben haben, vor dem Nutzen von Habitica produktiver gewesen zu sein, da man viel Zeit in das Verwalten von Aufgaben gesteckt investierte und folglich ein Verlust der Produktivität erfolgte. Infolgedessen kam ein ständiges Angstgefühl vor den Bestrafungen aus, die Eintreffen, sobald eine First nicht eingehalten werden konnte. Beim Öffnen der App und dem Anblick bestimmter Elemente darin entsteht der Drang, diese zu nutzen. Ohne diese visuelle Aufforderung würde der Gedanke, sie zu nutzen, nicht entstehen.
%\\
%Das abschließende Cluster handelt vom \enquote{blinden} Ausführen von Aufgaben, das Habitica durch seine Art fördert. Es gibt keine Unterscheidung zwischen mehr oder weniger sinnvollen Aufgaben im System. 
Der erste Abschnittskomplex zeigt, dass alle Teilnehmer mindestens einen der untersuchten kontraproduktiven Effekte (CPE) erlebt haben. Diese Effekte variieren in ihrer Häufigkeit und Schwere. Die am häufigsten auftretenden CPEs waren \enquote{\nameref{sec:cpe1}}, \enquote{\nameref{sec:cpe4}} und \enquote{\nameref{sec:cpe6}}. Es wurde festgestellt, dass das Erleben von CPEs ein allgemeines Problem ist, wobei es erhebliche Unterschiede in der Prävalenz und erlebten Ernsthaftigkeit der verschiedenen CPEs gibt.

Die Studie identifizierte auch weitere CPEs und ordnete sie in drei Cluster ein:
\begin{itemize}
	\item \textbf{Fehlende Flexibilität}: Dieses Cluster bezieht sich auf die fehlende Flexibilität von Habitica, um unterschiedliche Fälligkeitsdaten zuzuweisen oder das Abhaken von Aufgaben nach Mitternacht zu akzeptieren.
	\item \textbf{Subjektiver Verlust bei der Zielerreichung}: Einige Probanden gaben an, vor dem Nutzen von Habitica produktiver gewesen zu sein, da sie viel Zeit in das Verwalten von Aufgaben investierten, was zu einem Verlust der Produktivität führte. Dies führte zu einem ständigen Angstgefühl vor den Bestrafungen, die eintreffen würden, sobald eine Frist nicht eingehalten werden konnte.
	\item \textbf{Blinde Ausführung von Aufgaben}: Das letzte Cluster handelt vom “blinden” Ausführen von Aufgaben, das Habitica durch seine Art fördert. Es gibt keine Unterscheidung zwischen mehr oder weniger sinnvollen Aufgaben im System.
\end{itemize}

Es wurde festgestellt, dass das Öffnen der App und das Sehen bestimmter Elemente darin den Drang erzeugt, diese zu nutzen. Ohne diese visuelle Aufforderung würde der Gedanke, sie zu nutzen, nicht entstehen. Es wurde auch festgestellt, dass diejenigen, die das Belohnungssystem als unangemessen betrachten, dazu neigen könnten, kontraproduktive Nebeneffekte hervorzurufen.
\\
\\
\textbf{Korrelationen zwischen kontraproduktiven Effekten und Benutzererfahrung}
\\
%Es konnte festgestellt werden, dass der durchschnittliche wahrgenommene Nutzen und die durchschnittliche Zufriedenheit miteinander korrelieren. Beides hängt auch mit der Nutzungshäufigkeit zusammen. 
%Anders als in H2 angenommen, korrelieren durchschnittlich kontraproduktive Effekte nicht mit der Zufriedenheit und ebenso nicht mit den Nutzern. Wie in H3 angenommen, korrelieren metaperzeptive kontraproduktive Effekte hinsichtlich der Angemessenheit des Belohnungssystem mit der Zufriedenheit sowie dem Nutzen. Allerdings kann die Unangemessenheit des Betrrafungssystems nicht mit der Zufriedenheit korrelieren und auch nicht mit dem Nutzen.
\\
%Über den Verlauf der Spielphase nahm die Motivation Habitica zu verwenden, ab (Q5). In Bezug auf Q6 konnte keine signifikante Korrelation zwischen kontraproduktiven Effekten und Vor-Nutzungsmotivation oder der Nach-Nutzungsmotivation festgestellt werden. Die grundlegenden Korrelation von kontraproduktiven Effekten ist mit dem Verlust der Motivation verbunden. Zusammenfassend bedeutet das, je zufriedener Anwender waren, desto geringer war der Motivationsverlust und je mehr kontraproduktive Effekte sie erlebten (oder provozierten), desto geringer war der Motivationsverlust. Daher können kontraproduktive Effekte auch als eine Art Schutzfaktor gegen den Verlust der Motivation, Habitica zu spielen, auftreten. \\
Die Studie zeigt, dass die wahrgenommene Nützlichkeit und Zufriedenheit der Nutzer miteinander korrelieren und beide mit der Nutzungshäufigkeit zusammenhängen. Kontraproduktive Effekte korrelieren nicht generell mit Zufriedenheit und Nutzen, aber spezifische metaperzeptive kontraproduktive Effekte, die sich auf die Angemessenheit des Belohnungssystems beziehen, tun dies. Im Laufe der Zeit nahm die Motivation der Nutzer ab, Habitica zu verwenden. Es wurde keine signifikante Korrelation zwischen kontraproduktiven Effekten und der Motivation vor oder nach der Nutzung festgestellt. Allerdings wurde festgestellt, dass kontraproduktive Effekte mit einem geringeren Verlust an Motivation verbunden sind. Daher können kontraproduktive Effekte als eine Art Schutzfaktor gegen den Verlust der Motivation angesehen werden. Zusammenfassend lässt sich sagen, dass die Zufriedenheit der Nutzer und die Erfahrung kontraproduktiver Effekte beide dazu beitragen, den Verlust der Motivation zu minimieren.
\\
\textbf{Interindividuelle Unterschiede}
\\
Der letzte Fragekomplex zeigt, dass der Glaube an Gamifizierung positiv mit dem wahrgenommenen Nutzen und der Nutzungshäufigkeit korreliert. Die Korrelation zwischen dem Glauben an Gamifizierung und der Zufriedenheit war zwar positiv, aber nicht signifikant. Es wurde keine signifikante Korrelation zwischen dem Glauben an Gamifizierung und durchschnittlichen kontraproduktiven Effekten oder der Angemessenheit von Belohnungen festgestellt.

In Bezug auf die Benutzertypen waren der \enquote{Socializer} und der \enquote{Free Spirit} die am häufigsten dominanten Typen. Die anderen drei Spielertypen waren in der Testgruppe weniger dominant. Insgesamt scheint der Benutzertyp keinen relevanten Einfluss auf die Spielerfahrung in Habitica zu haben. Es wurden auch keine signifikanten Unterschiede im durchschnittlichen kontraproduktiven Effekt in Abhängigkeit vom dominanten Benutzertyp festgestellt. (vgl. \citep{diefenbach_counterproductive_2019})
%Wie in H4 angenommen, korreliert der Glauben an Gamifizierung positiv mit dem Nutzen und der Nutzungshäufigkeit (H4c). Die Korrelation zwischen dem Glauben und der Zufriedenheit war positiv, aber nicht signifikant (H4a). Es gibt keine signifikante Korrelation zwischen dem Glauben an Gamifizierung und dem durchschnittlichen kontraproduktiven Effekte und auch keine Korrelation zur Angemessenheit von Belohnungen (Q7).
%\\
%In Bezug auf Q8 ergab, dass der Socializer und der Free Spirit die am häufigsten dominanten Benutzertypen waren. Die anderen 3 Spielertypen waren nicht so dominant unter der Testgruppe vorhanden. Insgesamt scheint der Benutzertyp kein relevanter Faktor für die Spielerfahrung in Habitica zu sein. Zudem ergeben sich keine signifikanten Unterschiede im durchschnittlichen kontraproduktiven Effekt, abhängig vom dominanten Benutzertyp (vgl. \citep{diefenbach_counterproductive_2019}).

\subsection{Vorstellung des Projektmanagement Tools Asana}\label{sec:asana}
Dieser Abschnitt handelt von einer Zusammenfassung, welche Eigenschaften und Features Projektmanagement Tools im Allgemeinen haben sollen. Im Zusammenhang dieser Aufzählung erfolgt ein Review über das Projektmanagement Tool Asana.

\subsubsection{Basis-Features von Projektmanagement-Tools}
Die wichtigsten Features von Projektmanagement Tools werden im folgenden vorgestellt. (vgl. \cite{noauthor_projektmanagement_nodate})

\paragraph{Anforderungsmanagement}
Das Anforderungsmanagement umfasst zwei Hauptphasen. Die erste ist die Anforderungserhebung von einem Projekt. Dabei sollen initial die Anforderungen für ein Produkt oder ein Projekt erfasst werden. Dabei werden die Anforderungen sorgfältig dokumentiert, wobei die Kriterien der SMARTEn Formulierung entsprechen sollte. Die Anforderungen müssen grundlegende Bedingungen erfüllen.
Die zweite Hauptphase ist die Verwaltung der Anforderungen während des gesamten Projektlebenszyklus.
Dabei ist das Ziel, dass über die Projekt- oder Produktlaufzeit Anpassungen der Anforderungen ermöglicht werden. Es wird ein Überblick über die Anforderungen und genaue Kenntnis darüber gegeben, welche Bereiche angepasst werden müssen. ~\citep{venzmer_anforderungsmanagement-tool_2020}

\paragraph{Projektplanung}
Die wichtigsten Aufgaben bei der Projektplanung, die ein Projektmanagement Tool besitzen sollte, sind das Darstellen der Projektziele, die Planung der Aufgaben, eine zeitliche Dimension, ein Finanzierungsplan und die Kommunikation. Unter der Darstellung der Projektziele versteht man eine klare Definition, was mit einem Projekt erreicht werden soll und welchen Zeitrahmen für die Methoden zur Umsetzung festgelegt wird. Bei der Planung der Aufgaben ist es wichtig eine detaillierte Definition aller Arbeitspakete erstellen zu können. Dazu zählt ebenfalls die Strukturierung und Organisation. Die zeitliche Dimension legt den Zeitrahmen fest, bis wann ein Projekt fertig durchgeführt werden soll. Darunter fällt der Einsatz der klassischen Meilensteinplanung und die Visualisierung in einem Diagramm (beispielsweise einem Gantt-Diagramm). Zum Finanzierungsplan gehört die Verwendung von Stundensätzen und die Zeiterfassung, wie lang für eine Aufgabe Zeit aufgewendet wird. Innerhalb eines Projektes sollte es möglich sein Tools zu Kommunikation einzubinden oder, Funktionen haben, welche die Kommunikation zwischen allen Projektbeteiligten ermöglicht.~\citep{venzmer_projektplaner-software_2020};

\paragraph{Aufgabenverwaltung / Taskmanagement}
Task Management oder Aufgabenverwaltung bezeichnen das Planen, Organisieren, Überwachen und Umsetzen von Aufgaben. Es ist wichtig, einen klaren Überblick über seine Aufgaben zu behalten und sie effektiv zu erledigen. 
Die Verwaltung von Aufgaben beinhaltet die Priorisierung, Festlegung einer \enquote{Single Source of Truth} ohne Duplikate, die einheitliche Definition von Aufgaben und die kontinuierliche Pflege und Verwaltung. 
Die Priorisierung von Aufgaben hilft dabei Konflikte bei gleichzeitig auftretenden Aufgaben zu lösen.~\cite{venzmer_task_2020}

\paragraph{Ressourcenplanung}
Über Tools die eine Ressourcenplanung enthalten bieten eine Übersicht über die aktuelle Auslastung im Team oder Unternehmen. Das sind zum Beispiel Gantt Diagramme oder Portfolio Übersichten. Außerdem können so Möglichkeiten für Prognosen von zukünftigen Auslastungen in unterschiedlichsten Szenarien dargestellt werden.~\cite{venzmer_wie_2020}

\paragraph{Kommunikation und Zusammenarbeit}
Projektmanagement Tools sollten eine Möglichkeit der Kommunikation zwischen allen Projektbeteiligten beinhalten.

\paragraph{Projektcontrolling}
Projektcontrolling im Projektmanagement beinhaltet die Steuerung von Kosten und Aufwänden im Verhältnis zu den erzielten Ergebnissen. Es ist eine Querschnittsfunktion entlang des gesamten Projektlebenszyklus, es beinhaltet Kostenrechnungen, Monitoring, Reporting von Plan- und Ist-Werten sowie die Steuerung von Planabweichungen und Kapazitätsengpässen. Im klassischem Projektmanagement ist es eine kontinuierliche Aufgabe des Projektleiters, während es in agilen Projekten als integrierter Bestandteil in SCRUM-Terminen und Steuerungsmaßnahmen zu finden ist.~\cite{venzmer_projektcontrolling-tool_2020}

\paragraph{Dokumentenmanagement}
Dokumentmanagement bezeichnet das Verwalten von elektronischen Dokumenten über eine Datenbank, einschließlich der Digitalisierung von Papierdokumenten. Es umfasst das Erstellen, Bearbeiten, Archivieren und die Suche nach digitalen Dokumenten. Die Effizienzsteigerung im Arbeitsalltag durch schnellere Auffindbarkeit und Bearbeitung von Informationen sind Vorteile eines digitalen Dokumentmanagements.~\cite{schmuck_dokumentenmanagement_2019}

\paragraph{Multiprojektmanagement}
Im Multiprojektmanagement besteht das Ziel darin, die Auswahl, Planung, Steuerung und Kontrolle mehrerer Projekte zu organisieren, wobei eine effiziente Nutzung von Ressourcen (technisch, personell, finanziell) entscheidend ist. Die Priorisierung der Projekte im Bezug auf Unternehmens- oder Programmziele ist von großer Bedeutung. Eine eigene Abteilung für Multiprojektmanagement ermöglicht eine zentrale Steuerung aller Projekte und erleichtert die Identifizierung von Synergien zwischen den Vorhaben.~\cite{venzmer_multiprojektmanagement-software_2020}

\paragraph{Projektportfoliomanagement }
Projektportfoliomanagement bezeichnet die zentrale Verwaltung aller Projekte einer Organisation zur Erreichung strategischer Ziele. Es umfasst die Pflege und das Management der gesamten Projektwelt. Das Portfolio setzt sich aus Programmen und Einzelprojekten zusammen, wobei nicht bei allen Beteiligten die gleichen Ziele verfolgt werden müssen. Das Portfoliomanagement beinhaltet die fortlaufende Planung, Priorisierung, übergreifende Steuerung und das begleitende Controlling aller Projekte. Der Begriff und das Konzept des Projektportfoliomanagements haben sich seit der Veröffentlichung des \enquote{Standard for Portfolio-Management} durch das Project Management Institute (PMI) im Jahr 2008 verbreitet und ersetzen im deutschen Sprachraum den bisher gebräuchlichen Begriff \enquote{strategisches Multiprojektmanagement}. Die DIN 69909, eingeführt im Jahr 2013, verankert das Projektportfoliomanagement im deutschen Sprachraum und beschreibt Grundlagen, Prozessmodelle, Methoden und Rollen.~\cite{gmbh_detailseite_nodate}

\subsubsection{Funktionen von Asana}
Asana ist ein vielseitiges Arbeitsmanagement-Tool, das Einzelpersonen und Teams dabei unterstützt, Aufgaben zu verfolgen, Verantwortlichkeiten zu delegieren, den Fortschritt zu überwachen und in Echtzeit zu kommunizieren. Die Plattform dient als zentrale Anlaufstelle für die Zusammenarbeit, um Teams organisiert und fokussiert zu halten und sicherzustellen, dass Projekte termingerecht abgeschlossen werden.
\\

\paragraph{Koordination und Zusammenarbeit}
Asana erleichtert die Echtzeit-Koordination und Zusammenarbeit. Funktionen wie Multi-Homing und @-Erwähnung der Beteiligten ermöglichen eine effiziente Verfolgung des Fortschritts ohne Duplikation. Die Arbeitsbelastungsfunktion überwacht Teamkapazitäten, gleicht Verantwortlichkeiten aus und gewährleistet effiziente Arbeitsprozesse. Durch projektübergreifende Arbeitslastverfolgung können Ressourcen effizient verteilt werden, um die Zielerreichung zu fördern und Überlastung zu vermeiden.

\paragraph{Persönliche Produktivität}
Asana verbessert die individuelle Arbeitsverwaltung und steigert die Produktivität durch effizientes Organisations- und Zeitmanagement. \enquote{Meine Aufgaben} bieten eine zentrale Ansicht für die Verfolgung, Priorisierung und Statusüberwachung von zugewiesenen Aufgaben. Der Posteingang dient als zentrale Informationsquelle für Benachrichtigungen zu Aufgaben und Fortschritt, um Nutzer stets über relevante Änderungen auf dem Laufenden zu halten.

\paragraph{Transparenz und Verantwortung bei der Arbeit}
Durch Zuweisung von Aufgaben und Festlegung von Fälligkeitsdaten ermöglicht Asana eine klare Verantwortlichkeitszuweisung im Team. Die Plattform bietet Transparenz in den Arbeitsfortschritt und -status, erleichtert die Kommunikation und fördert eine effektive Zusammenarbeit.

\paragraph{Berichte und Ergebnisse teilen}
Asana fördert die Informationsgleichheit, indem es Projektfortschritte und Kennzahlen mit Vorgesetzten und Stakeholdern teilt. Kontinuierliche Statusaktualisierungen halten das Team informiert und optimieren Meetings. Asana Goals ermöglichen die klare Festlegung von Zielen und Schlüsselergebnissen, zentralisieren die Verfolgung, fördern die Ausrichtung des Teams auf gemeinsame Ziele und erleichtern Stakeholdern die Erfolgsberichterstattung.

\paragraph{Aufbau}
Asana organisiert die Arbeit auf mehreren Ebenen:
\begin{itemize}
\item \textbf{Ziele}: Übergeordnete Prioritäten für Unternehmen oder Teams, die klare Richtungen vorgeben und in Asana festgelegt sind.
\item \textbf{Portfolios}: Container, die zusammenhängende Projekte bündeln und den Fortschritt in Richtung bestimmter Initiativen oder Ziele verfolgen
\item \textbf{Projekte}: Organisieren und verwalten Arbeit im Zusammenhang mit spezifischen Initiativen. Projekte enthalten Aufgaben und Unteraufgaben, um die Arbeit in überschaubare Einheiten aufzuteilen und die Zusammenarbeit zu erleichtern.
\item \textbf{Aufgaben}: Einzelne Aktionspunkte innerhalb eines Projekts, mit Fälligkeitsdaten und Zuweisungen an Teammitglieder.
\item \textbf{Unteraufgaben}: Kleinere Arbeitseinheiten innerhalb einer Aufgabe, die oft genutzt werden, um komplexe Aufgaben in überschaubare Schritte aufzuteilen.
\end{itemize}
(vgl. \citep{noauthor_mit_nodate})

\subsubsection{Review zu Asana}
Wie folgender Artikel zeigt, beeinhaltet Asana alle oder einige der Aspekte~\cite{venzmer_projektmanagement_2020}. Zusammenfassend lässt sich sagen, dass das Task-Management das zentrale Merkmal von Asana ist. Es bietet vielfältige Möglichkeiten mit unterschiedlichen Darstellungen. Die Team-Zusammenarbeit wird effektiv durch Kommentare, Diskussionsbereiche und Benachrichtigungen unterstützt. Die Zeitleiste und der Workload für die Mitarbeiterauslastung ermöglicht eine übersichtliche Analyse bestehender Projekte und erlauben eine vorausschauende Planung, was in herkömmlichen Task-Management Systemen oft fehlt. Das Portfolio-Feature ermöglicht eine ganzheitliche Betrachtung des Unternehmensprogramms auf Projektebene.
Negativ hingegen ist der Datenschutz bei Asana ein wichtiges Anliegen, doch die Datenspeicherung findet, da es ein US-amerikanisches Land ist, nicht in Europa statt. Das könnte abschreckend auf potenzielle Kunden wirken. Intern werden verschiedene Daten für Analysen genutzt. Die 30-Tage Testversion bietet erweiterte Funktionen. jedoch erfolgt automatisch eine Umstellung auf einen kostenpflichtigen Account, es sei denn man kündigt diesen wiederholend vor Testlaufzeitende. Der volle Funktionsumfang ist bereits in der Basisversion verfügbar, was die Übersichtlichkeit beeinträchtigt. Es keinen Direktkontakt zum Support, es sein denn man besitzt einen Enterprise-Account.\\
Weitere Vorteile von Asana sind in diesem Artikel beschrieben~\cite{noauthor_asana_nodate}:
Asana zeichnet sich durch den besten kostenlosen Plan unter den überprüften Projektmanagement-Tools aus, während die kostenlosen Pläne anderer Alternativen oft für den Geschäftsgebrauch zu begrenzt sind. Die Plattform ist äußerst vielseitig, ermöglicht den Zugriff und die Verwaltung von Projekten und Aufgaben auf verschiedene Weisen, im Gegensatz zu manchen Programmen, die nur wenige Funktionen bieten.\\
Besonders herausragend ist die Automatisierungsfunktion von Asana, die die Optimierung wiederholter Aufgaben erleichtert, Zeit spart und die Produktivität steigert, indem manuelle Schritte vermieden werden.
Im Bezug auf Teamkommunikation bietet Asana hervorragende Funktionen, darunter die Möglichkeit, Prioritäten zu setzen, Aufgaben zuzuweisen und Kalender für die Planung zu nutzen. Dies ermöglicht eine effektive Zusammenarbeit, da Teams über Modifikationen und Entwicklungen auf dem Laufenden bleiben können. Asana kann auch zur Verfolgung von ToDo-Listen und Diskussionen im Projekt genutzt werden.
Negatives wurde in dem Artikel ebenfalls erwähnt. Die Nutzung von mobilen Geräten mit Asana ist eingeschränkt, obwohl jeder Plan eine kostenlose mobile App für iOS und Android enthält. Die Plattform funktioniert an Laptops oder Computern besser aufgrund der größeren Bildschirmgröße. Zudem erlaubt Asana nur die Zuweisung einer Aufgabe pro Benutzer. Für die Einbindung weiterer Benutzer in eine Aufgabe können Unteraufgaben verwendet oder für jede Aufgabe ein Mitarbeiter hinzugefügt werden.

\subsection{Lösungsansätze für Habitica}\label{sec:solutions_for_habitica}
In diesem Kapitel werden Lösungsvorschläge für die aus Kapitel \ref{sec:analysis} gesammelten \enquote{kontraproduktiven Nebeneffekte} anhand von Funktionen die Asana ausmachen konzeptionell ausgebessert oder gelöst.
Dabei werden auf die wichtigsten Nebeneffekte eingegangen, die in Studie 2 gemessen werden konnten. Außerdem werden die Cluster der zusätzlichen Nebeneffekte in Kombination mit den verbleibenden aus Studie 1 zusammengefasst eingegangen und Lösungsvorschläge konzipiert.
Die Bestrafung für Produktivität (vgl. \nameref{sec:cpe1}) ist eines der am häufigsten vorkommende Nebeneffekt, wie es in Studie 2 bewiesen werden konnte. \\
Zunächst ist es für den Anwender wichtig, dass er selbst entscheiden kann, bis zur welchen Uhrzeiten er seine Tasks abschließen möchte. Das ist wichtig, um zu vermeiden, dass der Tag beispielsweise um 23 Uhr und 59 Minuten endet. Somit kann er sein Fälligkeitsdatum genau definieren und ist nicht mehr von der Tageszeit des Tages innerhalb der 24 Stunden abhängig. Eine weitere Eindämmung dieser Problematik kann durch das Kapazitätsmanagement erfolgen. Wenn der Anwender jedem seiner Aufgaben einen Zeitaufwand mitgibt, kann er über die Statistik Tools einsehen, wie lange er für seine Aufgaben brauchen könnte und kann dadurch besser einschätzen, ob alle Aufgaben bis zum Fälligkeitsdatum absolviert werden können. Ein weiterer Ansatzpunkt ist der Umgang mit dem Abhaken der Aufgaben. Um ein Aufgabenboard idealerweise anzuwenden, ist es ratsam, absolvierte Aufgaben direkt nach dem erledigen abzuhaken. Sobald sich eine Routine entwickelt hat, schrumpft der Aufwand für die Strukturierung und Abhaken der Aufgaben.
Um nun den Gamifikation Ansatz hierbei zu verbessern, wäre es sinnvoll den Anwender nicht zu bestrafen, wenn er sich zu viel vorgenommen hat. In bekannten Videospielen wie bspw. \enquote{Assassin´s Creed} wir das Problem folgendermaßen gelöst: Der Spieler erhält nicht die volle Punktzahl oder Errungenschaft eines Levels, sofern er es nicht zu 100 Prozent gelöst hat. Etwas ähnliches ist hier denkbar. Der Anwender erhält folglich bspw. nur noch 70 Prozent der Belohnung.~\cite{noauthor_assassins_nodate}
\\
\\
Ein weiterer wichtiger Nebeneffekt, der auftritt, ist die Belohnung für irrelevante Aufgaben aus Kapitel \enquote{\nameref{sec:cpe4}}. 
Ohne eine kontrollierende Instanz ist dieser Nebeneffekt nur schwer zu lösen. Es benötigt hierbei eine Anpassung am Gamifiaktion Ansatz und einer Kontrolle darüber, welcher Aufwand eine erstellte Aufgabe wirklich enthält. Das wäre zum Beispiel eine Mindestanforderung an Unteraufgaben, die erledigt werden müssen und bspw. eine Maschine, die kontrolliert, ob die Aufgaben auch wirklich richtig abgeschlossen wurden. Grundlegend liegt hier allerdings ein falscher Nutzen der Anwendung vor. Der kann informativ abgedeckt oder vom System erkannt und darauf aufmerksam gemacht werden. Zusätzlich wäre es auch möglich, dass das System das schnelle Abhaken von Aufgaben blockiert und man es erst nach einer gewissen Zeit abhaken kann.
\\
\\
Die folgenden Nebeneffekte können in eine Kategorie eingeordnet werden, welche der vorangegangenen Kategorien eingeordnet werden, welche der vorangegangenen Kategorien fast entsprechen. Diese sind die \enquote{\nameref{sec:cpe6}} und \enquote{\nameref{sec:cpe7}}. Ähnlich wie im zuvor beschriebenen Fall ist es hier ebenso schwierig das lediglich durch das die zusätzliche Verwendung von Asana diese Nebeneffekte einzudämmen. Hierbei muss ebenso eine Überarbeitung des Gamifikation Ansatzes her, sodass dieser nicht mehr in der Häufigkeit auftreten kann, in der er durch Studie 2 nachweislich bewiesen auftritt. Denkbar ist es, wie bereits beschrieben, das Belohnungs-/ Bestrafungssystem umzugestalten, sodass die Bestrafung die Höhe der Belohnung einschränkt. So kann ein besserer Anreiz geschaffen werden, Aufgaben trotz Bestrafung in der Belohnung in der nachfolgenden Verwendung abzuschließen. Zusätzlich kann wie bereits erwähnt eine kontrollierende Einheit geschaffen werden, die darauf achtet, dass sinnvolle Aufgaben definiert werden und das Tool sinnvoll angewendet wird. Um einen Ersatz für die Gewohnheitsspalte von Habitica zu erhalten, ist es möglich Aufgaben als Dokumentationen zu erstellen, um darin zu definieren, welche Ziele der Anwender anstrebt und welche Veränderungen er dadurch bezwecken möchte. Er kann nun Unteraufgaben von diesen erstellen und in seinem Sprint Board integrieren und bei abgeschossener Tätigkeit abhaken. Hat er bestimmte Aufgaben nicht abgehakt, weil er die Gewohnheit nicht abstellen konnte, erhält er in diesem Fall keine Belohnung. Hinsichtlich der Motivation ist dies eine bessere Methode, als bestraft zu werden.
 \\
 \\
 In Studie 2 wurde zwar bewiesen, dass das Belohnungs-/ Bestrafungssystem als angemessen betrachtet werden konnte, allerdings konnte ebenso bewiesen werden, dass das System und die Meta-Wahrnehmungen der Nebeneffekte korrelieren. Das hat zur Ursache, dass konzeptionell das Belohnugs-/ Bestrafungssystem eine Überarbeitung benötigt. In den zuvor 3 behandelten Lösungsvorschläge für die wichtigsten erschienen Nebeneffekte, wurde bereits angerissen, wie es verbessert werden kann. Zum einen kann die Art der Bestrafung herausgenommen werden und Teil der Quests sein, dass der Avatar des Anwenders nicht durch nicht Erledigen von Aufgaben Leben verliert. Dies soll durch das Absolvieren von Quests ersetzt werden. Zum anderen wäre es sinnvoll, dass die Belohnung bspw. bei nicht pünktlichen Abhaken von Aufgaben nicht den vollen Umfang enthält, sondern nur noch 80 oder 70 Prozent der eigentlichen Belohnung. Zusätzlich dazu, ist es sinnvoll den Umfang der Belohnungen an den Aufwand von Aufgaben anzupassen. Dieser Umstand wurde in den Nebeneffekten \enquote{\nameref{sec:acpe1}} und \enquote{\nameref{sec:acpe2}} erwähnt. Für den Anwender sollte das System für das Erledigen eine in der Höhe sinnvolle Belohnung erhalten. Das sollte aber nicht nur materiell sein, also Goldmünzen und Mana, sondern auch Ausrüstungsgegenstände, die bei der derzeitigen oder weiteren Quests wichtig sind, oder Quests, die neue Geschichten freischalten können oder weitere Abenteuer mit sich bringen können.
 \\
 \\
Aus Studie 2 gehen 3 neue Cluster an Nebeneffekten hervor, welche nun in Verbindung mit den Nebeneffekten aus Studie 1 abgehandelt werden.
Das erste Cluster bezog sich auf die fehlende Flexibilität von Habitica. Darunter fallen zusätzliche die Aspekte \enquote{\nameref{sec:sr2}} und \enquote{\nameref{sec:sr3}}. Asana ist in seiner Art ziemlich benutzerdefiniert Anpassbar. Der Anwender von Asana kann nach Belieben den Aufbau der Benutzeroberfläche gestalten. Bspw. kann er ein Projektboard anlegen, welches als Backlog dienen soll. Für seine Wochenaufgaben kann er nun weitere Boards anlegen, in welchen er seine Aufgaben verlinkt. Im Backlog kann er dann Automatismen einbauen, durch welche zum Beispiel abgearbeitete Aufgaben in einen bestimmten Bereich geschoben werden. Außerdem ist es in Asana ebenfalls möglich sogenannte \enquote{Third-Party-Tools} einzubinden, die dem Anwender zusätzlich von Nutzen sein können. Außerdem besitzt Asana ein großes Nachschlage Lexikon, in welchem viele Artikel zu bestimmten Handhabungen in Asana erklärt werden und Ratgeber, wie die Arbeit mit dem Projektboard optimiert werden kann.
\\
\\
Das zweite Cluster bezog sich auf den subjektiven Verlust beim Erreichen von Zielen und das Gefühl, dass der verlorenen Produktivität und die Erinnerung an Laster. Dazu zählt der Aspekt \enquote{\nameref{sec:sr1}} aus Studie 1.
Um einen Überblick über seine selbst gesteckten Ziele zu erreichen, gibt es in Asana z.B. die Möglichkeit Berichte über den Fortschritt zu erhalten, wie viele Aufgaben, die zu einem Ziel gehören, erreicht wurden. Wie bereits im vorherigen Aspekt bezüglich der risikofreien Belohnungen erwähnt, kann es mehr von Vorteil sein, wenn gesammelte Laster in einer anderen Darstellungsform in einem anderen Projektboard definiert werden können. Hierbei sollen lediglich einzelne Aufgaben, die zu diesen Laster gehören, vom Nutzer selbst ausgewählt und im Sprint Board hinzugefügt werden. Somit wären sie im Hauptprojekt visuell nicht mehr sichtbar und außerhalb des Aufmerksamkeitsradius des Anwenders.
\\
\\
Das letzte Cluster bezieht sich auf das \enquote{blinde} Abarbeiten von Aufgaben.
Wie bereits erwähnt, kann hiervon von Nutzen sein, gesammelte Aufgaben in das Backlog zu verschieben, sodass sie nicht im gleichen Board, wie die Wochenaufgaben mit hoher Priorität aufgeführt werden. Über das Kapazitätsmanagement erhält der Anwender eine Übersicht darüber, ob es zeitlich sinnvoll ist noch weitere Aufgaben zu erledigen oder zu seiner Sprint-Woche hinzuzufügen. Außerdem helfen hierbei auch die Ratgeber von Asana, wie man eine sinnvolle Planphase innerhalb des Backlogs ausführt, und wie man die am besten geeigneten Aufgaben in seine Wöchentlichen Aufgaben einplant.

\section{Fazit und Ausblick}\label{sec:fazit}
Asana ist ein sehr benutzerdefiniert gestaltbares Projektmanagement Tool, das in einigen Aspekten die auftretenden Nebeneffekte konzeptionell eindämmen könnte, die in Habitica zum Vorschein kommen. Allerdings konnten nicht alle Aspekte aufgegriffen werden, da sich einige Effekte auf die Implementierung der Gamification beziehen, welche im Rahmen dieser Ausarbeitung nicht gelöst werden können. Zusammengefasst ist es also möglich, die Forschungsfrage konzeptionell hinreichend zu beantworten, dass Habitica durch eine Hinzunahme eines Projektmanagement Tools wie Asana verbessert werden kann. Um zu beweisen, dass diese Effekte nun nicht mehr auftreten, müssten die Aspekte umgesetzt werden und anschließend qualitativ und quantitativ gemessen werden. Habitica ist in seiner Implementierung seines Boards und seiner Statistiken zu unausgereift, dass ein Anwender einen richtigen Nutzen daraus ziehen kann. Durch die mangelnde Flexibilität des Systems, kann dieser sein Potenzial nicht vollkommen ausschöpfen. Außerdem gibt Habitica dem Anwender zu wenig Hilfestellungen über die richtige Bedienung der Anwendung, welche beispielsweise in Asana in großen Umfang gegeben ist.
Abschließend lässt sich feststellen, dass die praktische Umsetzung der Erkenntnisse dieser Ausarbeitung realistisch erscheint und bei entsprechender zeitlicher Verfügbarkeit in Form einer neu konzipierten Anwendung initiiert werden kann.


\bibliographystyle{ACM-Reference-Format}
\bibliography{sample}

\end{document}
\endinput

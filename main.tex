% Holodecks template version of 2023-09-01 enhances the ACM template, version 1.7.0:
% https://www.acm.org/publications/proceedings-template
% The ACM Latex guide provides further information about the ACM template

\documentclass[sigconf, nonacm]{acmart}

%% The following content must be adapted for the final version
% paper-specific
\newcommand\holodoi{XX.XX/XXX.XX}
\newcommand\holopages{XXX-XXX}
% issue-specific
\newcommand\holovolume{1}
\newcommand\holoissue{1}
\newcommand\holoyear{2023}
% should be fine as it is
\newcommand\holoauthors{\authors}
\newcommand\holotitle{\shorttitle} 
% leave empty if no availability url should be set
\newcommand\holoavailabilityurl{URL_TO_YOUR_ARTIFACTS}
% whether page numbers should be shown or not, use 'plain' for review versions, 'empty' for camera ready
\newcommand\holopagestyle{plain} 

\begin{document}
\title{Konzeption eines Projektmanagement Task-Boards unter Anwendung des Habitica Gamification-Ansatzes}

%%
%% The "author" command and its associated commands are used to define the authors and their affiliations.
\author{Nick Philipp Häcker}
\affiliation{%
  \institution{Fakultät für Digitale Medien}
  \institution{Hochschule Furtwangen University}
}
\email{nick.athaeck@gmail.com}




%%
%% The abstract is a short summary of the work to be presented in the
%% article.
\begin{abstract}
In der heutigen Welt stehen viele sowohl Studierende als auch Werktätige vor einer Vielzahl von Herausforderungen, um ihre Produktivität aufrechtzuerhalten. Um das zu erreichen gibt es viele verschiedene Möglichkeiten. Dabei werden Aufgaben häufig in Papier Format aufgeschrieben, oder man verwendet Tools wie Notion, GitHub Projects oder Habitica. 
Diese Arbeit beschäftigt sich mit dem Gamification-Ansatz von Habitica, sammelt seine Schwachstellen und konzipiert Lösungen für eine bessere Integration der Spielerischen Elemente. Dabei wird in der Konzeption des Aufgaben-Boards der Aufbau und die Funktionsweise eines gestaltbaren Projektmanagements Tools, wie er in Asana vorzufinden ist, miteinbezogen und angewandt.

\end{abstract}

\maketitle

%%% do not modify the following Holodeck block %%
%%% Holodeck block start %%%
\pagestyle{\holopagestyle}
\begingroup\small\noindent\raggedright\textbf{CCS Concepts:}\\
CCS $\rightarrow$ Human-centered computing $\rightarrow$ Human computer interaction (HCI) $\rightarrow$ HCI design and evaluation methods $\rightarrow$ User studies;
CCS $\rightarrow$ Theory of computation $\rightarrow$ Theory and algorithms for application domains $\rightarrow$ Algorithmic game theory and mechanism design $\rightarrow$ Network games; CCS $\rightarrow$ Software and its engineering $\rightarrow$ Software organization and properties $\rightarrow$ Contextual software domains $\rightarrow$ Virtual worlds software $\rightarrow$ Interactive games


 \renewcommand\thefootnote{}\footnote{\noindent
	Seminar Interaktionsdesign \\
	\emph{Interaktionsdesign MIM1}, WS 2023/2024\\
	Dozent: Prof. Dr. Thomas Schlegel\\
	Modul: Interaktionsdesign, Fakultät Digitale Medien
 \\
 }\addtocounter{footnote}{-1}\endgroup
%%% Holodecks block end %%%

%%% do not modify the following Holodecks block %%
%%% Holodecks block start %%%
\ifdefempty{\holoavailabilityurl}{}{
\vspace{.3cm}
\begingroup\small\noindent\raggedright\textbf{Keywords}\\
Gamification, Asana, 3D Environment
% The source code, data, and/or other artifacts have been made available at \url{\holoavailabilityurl}.
\endgroup
}
%%% Holodecks block end %%%

\section{Einleitung}



\subsection{Aufbau des Papers}

\section{Stand der Forschung}

\subsection{Gamification}

\subsection{Projektmanagement Tools}

\section{Ansatz der Konzeption}

\subsection{Analyse von Habitica}

\subsection{Analyse von Asana}

\section{Fazit und Ausblick}


\bibliographystyle{ACM-Reference-Format}
\bibliography{sample}

\end{document}
\endinput

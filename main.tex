% Holodecks template version of 2023-09-01 enhances the ACM template, version 1.7.0:
% https://www.acm.org/publications/proceedings-template
% The ACM Latex guide provides further information about the ACM template

\documentclass[sigconf, nonacm]{acmart}
\usepackage[german]{babel}
\usepackage{csquotes}
%% The following content must be adapted for the final version
% paper-specific
\newcommand\holodoi{XX.XX/XXX.XX}
\newcommand\holopages{XXX-XXX}
% issue-specific
\newcommand\holovolume{1}
\newcommand\holoissue{1}
\newcommand\holoyear{2023}
% should be fine as it is
\newcommand\holoauthors{\authors}
\newcommand\holotitle{\shorttitle} 
% leave empty if no availability url should be set
\newcommand\holoavailabilityurl{URL_TO_YOUR_ARTIFACTS}
% whether page numbers should be shown or not, use 'plain' for review versions, 'empty' for camera ready
\newcommand\holopagestyle{plain} 

\begin{document}
\title{Konzeption eines Projektmanagement Task-Boards unter Anwendung des Habitica Gamification-Ansatzes}

%%
%% The "author" command and its associated commands are used to define the authors and their affiliations.
\author{Nick Philipp Häcker}
\affiliation{%
  \institution{Fakultät für Digitale Medien}
  \institution{Hochschule Furtwangen University}
}
\email{nick.athaeck@gmail.com}




%%
%% The abstract is a short summary of the work to be presented in the
%% article.
\begin{abstract}
In der heutigen Welt stehen viele sowohl Studierende als auch Werktätige vor einer Vielzahl von Herausforderungen, um ihre Produktivität aufrechtzuerhalten. Um das zu erreichen gibt es viele verschiedene Möglichkeiten. Dabei werden Aufgaben häufig in Papier Format aufgeschrieben, oder es werden Tools wie Notion, GitHub Projects oder Habitica verwendet. 
Diese Arbeit beschäftigt sich mit dem Gamification-Ansatz von Habitica, sammelt seine Schwachstellen und konzipiert Lösungen für eine bessere Integration der Spielerischen Elemente. Dabei wird in der Konzeption des Aufgaben-Boards der Aufbau und die Funktionsweise eines gestaltbaren Projektmanagements Tools, wie er in Asana vorzufinden ist, miteinbezogen und angewandt.\\
(nochmal überarbeiten um das gesamte Paper am Ende zusammenfassen zu können)
\end{abstract}

\maketitle

%%% do not modify the following Holodeck block %%
%%% Holodeck block start %%%
\pagestyle{\holopagestyle}
\begingroup\small\noindent\raggedright\textbf{CCS Concepts:}\\
CCS $\rightarrow$ Human-centered computing $\rightarrow$ Human computer interaction (HCI) $\rightarrow$ HCI design and evaluation methods $\rightarrow$ User studies;
CCS $\rightarrow$ Theory of computation $\rightarrow$ Theory and algorithms for application domains $\rightarrow$ Algorithmic game theory and mechanism design $\rightarrow$ Network games; CCS $\rightarrow$ Software and its engineering $\rightarrow$ Software organization and properties $\rightarrow$ Contextual software domains $\rightarrow$ Virtual worlds software $\rightarrow$ Interactive games


 \renewcommand\thefootnote{}\footnote{\noindent
	Seminar Interaktionsdesign \\
	\emph{Interaktionsdesign MIM1}, WS 2023/2024\\
	Dozent: Prof. Dr. Thomas Schlegel\\
	Modul: Interaktionsdesign, Fakultät Digitale Medien
 \\
 }\addtocounter{footnote}{-1}\endgroup
%%% Holodecks block end %%%

%%% do not modify the following Holodecks block %%
%%% Holodecks block start %%%
\ifdefempty{\holoavailabilityurl}{}{
\vspace{.3cm}
\begingroup\small\noindent\raggedright\textbf{Keywords}\\
Gamification, Asana, 3D Environment, Task-Management
% The source code, data, and/or other artifacts have been made available at \url{\holoavailabilityurl}.
\endgroup
}
%%% Holodecks block end %%%

\section{Einleitung}
Das in der Veranstaltung \enquote{Interaktionsdesigin} entstandene Projekt der Hochschule Furtwangen dient als technisches \enquote{prove-of-concept} um eine Anwendung, wie sie hier konzipiert wird, umzusetzen.  \\
Ein bevorzugtes Mittel um seine Tages-, Wochen- oder gar Monatsaufgaben zu sammeln und zu planen, greifen viele Menschen zu Stift und einem Blatt Papier oder einem Kalender und schreiben sich Termine und/ oder Aufgaben auf. Für die Einen dient ein Kalender und ein Paper oder Block als ToDo-Liste, für die Anderen dienen dabei digitale Anwendungen als Hilfsmittel für die Aufgabenplanung.\\
Um einen Projektplaner nicht nur Monoton und dem Nutzen zum Zweck zu gestalten, gab es die Idee Gamification-Elemente einzubinden um eine Mischung aus einem ToDo-Board und einem kleinem "Role-Playing-Game" (RPG) zu erhalten. Dieser Idee ist bereits ein Entwicklerteam gefolgt und haben eine Anwendung gebaut, die diese 2 Aspekte vereint. Es dient zusätzlich als ein "Habit-Tracker", um den eigenen Lebensstil anzupassen. Dieses Paper bedient sich dabei an einer Forschung der Ludwig-Maximilians-Universität, München, Fakultät der Psychologie mit dem Namen \enquote{Counterproductive effetcs of gamification: An analysis \allowbreak on the example of the gamified task manager Habitica}. Dabei wurden bei einer Nutzerstudie von 45 Probanden Kontraproduktive Nebeneffekte festgestellt, welche in der Anwendung Habitica festgestellt wurde. Das Ziel ist es, diese Nebeneffekte zu sammeln und Lösungen zu konzipieren, die in einer neuen Form dieser Anwendung umgesetzt werden sollen. Dabei werden bestehende Projektmanagement-Tools und Gamification Ansätze betrachtet, die für die Konzeption relevant sind. Die Konzeption orientiert sich dabei am User-Centered-Design-Prozess nach ISO 9245-210 und wird in Form eines abgewandelten Nutzertests evaluiert..\\
Die Forschungsfrage dieses Paper lautet daher:\\
\textbf{Wie können paradoxe Nebeneffekte der Bedienung \allowbreak von Habitica durch das Miteinbeziehen eines Projektmanagement Tools wie Asana vermindert werden?}
\\
\\
\\
(Hier Struktur des Papers einbauen, sobald alle Kapitel geschrieben sind)
\section{Grundlagen}
\subsection{Gamification}

\subsubsection{Gamification Elemente}

\subsubsection{Spielertypen nach Bartle}

\subsubsection{Motivation}

\subsection{Projektmanagement und Tools im Allgemeinen}
Das Wort Projektmanagement besteht aus den zwei Wörtern \enquote{Projekt} und \enquote{Management}. 
In diesem Kontext wird ein Projekt nach der DIN 69901-5 als eine \enquote{Absicht, die im Wesentlichen durch die Einzigartigkeit der Bedienung in ihrer Gesamtheit gekennzeichnet ist} Es zeichnet sich durch eine Zielvorgabe mit zeitlichen, finanziellen und personellen Einschränkungen aus.~\cite{DIN69901-5} 

\subsubsection{Aufbau}

\subsubsection{Ziele}

\subsection{Habitica}

\section{Ansatz der Konzeption}

\subsection{Analyse von Habitica}

\subsection{Analyse von Asana}

\subsection{Lösungsansätze für Habitica}

\section{Fazit und Ausblick}

%Some examples of references. A paginated journal article~\cite{Abril07}, an enumerated journal article~\cite{Cohen07}, a reference to an entire issue~\cite{JCohen96}, a monograph (whole book) ~\cite{Kosiur01}, a monograph/whole book in a series (see 2a in spec. document)~\cite{Harel79}, a divisible-book such as an anthology or compilation~\cite{Editor00} followed by the same example, however we only output the series if the volume number is given~\cite{Editor00a} (so Editor00a's series should NOT be present since it has no vol. no.), a chapter in a divisible book~\cite{Spector90}, a chapter in a divisible book in a series~\cite{Douglass98}, a multi-volume work as book~\cite{Knuth97}, an article in a proceedings (of a conference, symposium, workshop for example) (paginated proceedings article)~\cite{Andler79}, a proceedings article with all possible elements~\cite{Smith10}, an example of an enumerated proceedings article~\cite{VanGundy07}, an informally published work~\cite{Harel78}, a doctoral dissertation~\cite{Clarkson85}, a master's thesis~\cite{anisi03}, an finally two online documents or world wide web resources~\cite{Thornburg01, Ablamowicz07}.

\bibliographystyle{ACM-Reference-Format}
\bibliography{sample}

\end{document}
\endinput
